\chapter{Przegląd literatury}
\label{ch:literatura}
Rozdział przedstawia przegląd literatury dotyczącej prognozowania cen energii elektrycznej, ze szczególnym uwzględnieniem zmiennych wejściowych oraz zwykle używanych modeli prognozowania. W analizie odwołano się zarówno do prac profesora Rafała Werona z Politechniki Wrocławskiej, który jest uznanym ekspertem w dziedzinie EPF, jak i do badań innych autorów, aby zapewnić kompleksowy kontekst dla przeprowadzonego badania.

\section{Zmienne wejściowe}
\label{sec:zmienne_wejsciowe_literatura}

W przeglądzie literatury dotyczącym prognozowania cen energii elektrycznej na rynkach dnia następnego, artykuł napisany przez Jesus Lago o współautorstwie prof. Werona \cite{LAGO2021116983} może stanowić istotny punkt odniesienia pod względem wyboru zmiennych wejściowych. Autorzy przyjęli zestaw dostępnych danych wejściowych dla prognozowania godzinowych cen na rynkach energii elektrycznej Nord Pool, EPEX-BE, EPEX-FR, EPEX-DE oraz PJM.
Niezależnie od zastosowanych przez autorów modeli (LEAR czy DNN), dla wszystkich rynków stworzono wektory zawierające po 24 wartości odpowiadające każdej godzinie dla cen z poprzednich trzech dni oraz tygodnia wstecz. Do tych wektorów dodano wektor określający dzień tygodnia w celu uwzględnienia sezonowości dziennej. Dodatkowo pod uwagę wzięto zmienne fundamentalne, różniące się w zależności od analizowanego rynku. \newline
Dla rynku Nord Pool (NP) były to prognozy zapotrzebowania na moc oraz generacji energii wiatrowej.\newline
Dla rynku PJM w Stanach Zjednoczonych uwzględniono dwie serie prognoz zapotrzebowania na moc: prognozę zapotrzebowania dla całego systemu oraz prognozę dla strefy Commonwealth Edison.\newline
Dla rynku EPEX-BE w Belgii wykorzystano prognozę zapotrzebowania na moc we Francji oraz prognozę generacji we Francji, ponieważ wcześniejsze badania wykazują, że właśnie te dwie zmienne są najlepszymi predyktorami cen belgijskich. \newline
Dla rynku EPEX-FR we Francji brano pod uwagę prognozę zapotrzebowania na moc oraz prognozę generacji.\newline
Dla rynku EPEX-DE w Niemczech uwzględniono prognozę zapotrzebowania na moc oraz zagregowane prognozy generacji energii wiatrowej i słonecznej.\newline
Do wszystkich wymienionych zmiennych fundamentalnych dodano wektory godzinowe tych zmiennych na dzień oraz tydzień wstecz.

Kolejnym artykułem wartym uwagi jest artykuł skupiony na postprocesingu \cite{LIPIECKI2024107934} w prognozowaniu cen energii elektrycznej metodami probabilistycznymi. Do osiągnięcia celu zostały wykorzystane następujące zmienne: 
\begin{itemize}
    \item Historyczne ceny energii elektrycznej z poprzednich jednego, dwóch, trzech i siedmiu dni
    \item Prognoza systemowego zapotrzebowania na moc
    \item Prognoza generacji energii z odnawialnych źródeł będąca sumą prognoz generacji energii wiatrowej i słonecznej
    \item Ceny uprawnień do emisji dwutlenku węgla
    \item Ceny gazu ziemnego
    \item Ceny ropy naftowej Brent
    \item Cena emisji węgla
    \item Zmienne czasowe reprezentujące dni tygodnia
    \item Zmienne czasowe, czyli zmienne binarne reprezentujące dni tygodnia
\end{itemize}
Warto podkreślić, że autorzy nie używali surowych danych do modelowania, lecz zastosowali transformację hiperbolicznego sinusa do zmiennych wyjściowych w celu stabilizacji wariancji przed modelowaniem. Transformacja ta jest opisana wzorem:
\[
\sinh(x) = \frac{e^x - e^{-x}}{2}
\]
gdzie \( e \) to liczba Eulera.

W innym artykule o współautorstwie prof. Werona \cite{en9080621} podkreśla się, że opóźnione ceny oraz zmienne określające obciążenie systemowe oraz strefowe, czyli zapotrzebowanie na moc w poszczególnych strefach, bądź w całym systemie energetycznym w stosunku do całej dostępnej mocy, mają bardzo istotny wpływ na prognozowanie cen. Korzystając ze zbiorów danych z rynków Nord Pool oraz rynku Brytyjskiego, autorzy użyli używali tych zmiennych do prognozowania cen energii elektrycznej.

Artykuł prof. Ziel \cite{ZIEL201598} analizujący niemiecki RDN, potwierdza poprzednie tezy, wskazując na znaczenie cen energii opóźnionych, obciążenia sieci i energii z OZE. Opisujący przez niego model autoregresyjny opisuje każdą z tych zmiennych jako zależną od jej własnych opóźnionych wartości. Wprowadza skomplikowany zestaw cech opóźnionych jak i również opóźnione zależności niektórych z cech:
\begin{itemize}
    \item Opóźnienia cen energii elektrycznej o 1, 2, …, 361, 504, 505, 672, 673, 840, 841, 1008, 1009 godzin
    \item Opóźnienia obciążenia sieci o 1, 2, …, 361, 504, 505, 672, 673, 840, 841, 1008, 1009 godzin
    \item Opóźnienia generacji energii z OZE o 1, 2, …, 361 godzine
    \item Opóźnienia obciążenia zależnego od ceny
    \item Opóźnienia ceny zależnej od obciążenia o 1, 2, …, 361, 504, 505, 672, 673, 840, 841, 1008, 1009 godzin
    \item Opóźnienia ceny zależnej od generacji energii z OZE o 1, 2, …, 49 godzin
    \item Opóźnienia obciążenia zależnego od generacji energii z OZE o 1, 2, …, 49 godzin
\end{itemize}
Oprócz opóźnionych wartości zmiennych, model uwzględnia również trend i efekty sezonowe, takie jak efekty czasowe i kalendarzowe, godziny w ciągu dnia, dni tygodnia, święta publiczne (w tym krajowe i regionalne), zmiany czasu letniego.

Innym przykładem jest praca inżynierska napisana przez Mikołaj Kalisz i Adam Mantiuk na Politechnice Warszawskiej \cite{MGR2025}. Analizując rynek energii elektrycznej w Polsce w latach 2021-2023, autorzy analizują zmienne średnią temperaturę godzinową w Polsce, handel międzynarodowy, kurs polskiego złotego względem euro oraz dolara oraz poziom inflacji miesiąc do miesiąca i rok do roku oraz rezerwy mocy ponad i poniżej zapotrzebowania. Po analizie korelacji z wielkiego zestawu cech wybierają dziesięć zmiennych, które mają największy wpływ na cenę energii w wybranym okresie. Są to: produkcja energii z fotowoltaiki, rezerwa mocy poniżej zapotrzebowania, cena uprawnień emisyjnych (CO2), inflacja w porównaniu do poprzedniego roku, cena gazu oraz historyczne ceny prądu opóźnione od dwóch do sześciu dni. 

Podsumowując, w literaturze dotyczącej prognozowania cen energii elektrycznej na RDN zmienne wejściowe są różnorodne i zależą od specyfiki danego rynku. Wybrane badania podkreślają znaczenie historycznych cen energii, zapotrzebowanie  oraz generacji energii z OZE jako kluczowych zmiennych wpływających na prognozy. Dodatkowo, zmienne fundamentalne, takie jak ceny surowców energetycznych czy zmienne czasowe, również odgrywają istotną rolę. Wszystkie z kluczowych cech są wykorzystywane w niniejszej pracy.

\section{Modele prognozowania}
\label{sec:modele_prognozowania_literatura}

Wśród stosowanych metod do predykcji cen energii elektrycznej wyróżnia się różne podejścia. Rafał Weron w swojej ponadczasowej pracy z 2014 roku "Electricity price forecasting: A review of the state-of-the-art with a look into the future" \cite{WERON20141030} dokonuje przeglądu literatury z ubiegłych wtedy 15 lat, systematyzując szybko rosnącą liczbę publikacji w tej dziedzinie. Weron wyjaśnia mechanizmy kształtowania się cen na rynkach energii elektrycznej, koncentrując się na cenach dnia następnego. Klasyfikuje techniki predykcyjne pod względem horyzontu czasowego i zastosowanej metodologii. Wymienia następujące kategorie modeli: 
\begin{itemize}
    \item multi-agent
    \item fundamentalne,
    \item reduced-form,
    \item statystyczne,
    \item computational intelligence,
    \item hybrydowe.
\end{itemize}

Modele multi-agent symulują zachowanie uczestników rynku energii, takich jak producenci i konsumenci, w celu przewidywania cen. Weron \cite{WERON20141030} wskazuje, że modele oparte na równowadze Nasha-Cournota czy równowadze funkcji podaży, są przydatne w analizie długoterminowej, ale nie uwzględniają strategiczne obstawianie cen przez uczestników rynku. W związku z tym tego rodzaju metody pasują najbardziej do stabilnych rynków bez dużych wahań cenowych. 

Modele fundamentalne opierają się na analizie czynników ekonomicznych i fizycznych, takich jak ceny paliw, emisje CO2 czy zapotrzebowanie. Weron \cite{WERON20141030} dzieli je na parameter-rich (uwzględniające wiele zmiennych) i parsimonious structural (uproszczone). Głównymi wyzwaniami takich modeli są ich złożoność oraz duże ilości danych, które często mogą być niedostępne w czasie rzeczywistym. 

Modele reduced-form opisują dynamikę cen za pomocą procesów stochastycznych, takich jak jump-diffusions czy Markov regime-switching. Weron \cite{WERON20141030} wskazuje, że są one użyteczne w modelowaniu dziennej zmienności cen, ale mogą nie być dokładne w próbie dokładnego liczenia cen godzinowych, gdyż nie uwględniają wpływu zmiennych fundamentalnych, takich jak sezonowość czy zmiany w podaży i popycie.

Modele statystyczne, takie jak ARIMA i GARCH, są szeroko stosowane w krótkoterminowym EPF. Modele autoregresyjne (AR) wykorzystują liniową kombinację przeszłych wartości zmiennej do prognozowania przyszłych wartości. Modele średniej ruchomej (MA) prognozują zmienną na podstawie liniowej kombinacji przeszłych błędów prognoz. Modele ARMA łączy te dwa podejścia. Modele ARIMA dodają różnicowanie, co pozwala uwzględniać niestacjonarność danych, czyli ich zmienność w czasie. W celu uwzględnienia sezonowości wykorzystuje się również model SARIMA. Zgodnie z \cite{appliedmath3020018} modele SARIMA wykazują dobre wyniki w prognozowaniu cen energii elektrycznej, ale ich skuteczność może być ograniczona w przypadku danych o dużej zmienności. Weron \cite{WERON20141030} wspomina również o modelach ARX i ARMAX, gdzie symbol X oznacza zmienne zewnętrzne, takie jak pogoda czy zapotrzebowanie, które pomagają wyjaśnić zmiany cen. Popularny równiez w literaturze Model GARCH jest szczególnie przydatny w modelowaniu zmienności cen energii elektrycznej, ponieważ potrafi uwzględniać nierównomierne wahania cen (heteroskedastyczność) oraz fakt, że te wahania zmieniają się w czasie, zależąc od wcześniejszych zmian cen i ich niestabilności.

Modele computational intelligence, oparte na technikach uczenia maszynowego, zyskały na popularności w nowszych badaniach. Wśród popularnych metod wyróżniają się metody rozmyte, metody wektorów nośnych, LSTM oraz CNN. Jednym z przykładów zastosowania takiego modelu jest wspomniana przeze mnie praca inżynierska na Politechnice Warszawskiej \cite{MGR2025}. W celu stworzenia modelu do skomplikowanego okresu rynkowego od 2021 do 2023 roku, autorzy używają Perceptronu wielowarstwowego (MLP). W innej pracy Grzegorza Marcjasza \cite{en13184605}, proponowany jest dobór hiperparametrów do głębokiej sieci neuronowej (DNN). Badanie porównuje DNN do statystycznego modelu LASSO i podkreśla, iż wyniki DNN są lepsze, co może wskazywać na przewagę nowoczesnego uczenia maszynowego nad tradycyjnymi metodami statystycznymi.

Modele hybrydowe łączą elementy różnych kategorii, aby wykorzystać ich zalety. Jinliang Zhang jest przykładem takiego podejścia. W pracy \cite{TAN20103606} łączy transformację falkową z ARIMA i GARCH. Argumentuje to w sposób, że połączenie WT z modelami ARIMA i GARCH pozwala na skuteczne modelowanie złożonych cech cen energii elektrycznej, takich jak niestacjonarność, nieliniowość i wysoka zmienność. W późniejszej pracy Zhang \cite{ZHANG2012695} łączy wspomniane transformację falkową oraz ARIMA z metodą najmniejszych kwadratów maszyn wektorów nośnych (LSSVM). Potwierdzając skuteczność takich metód, Weron \cite{WERON20141030} podaje, że popularnym podejściem hybrydowym jest łączenie modeli statystycznych z sieciami neuronowymi, co pozwala na modelowanie zarówno liniowych, jak i nieliniowych zależności.

W niniejszej pracy stworzony zbiór danych jest analizowany za pomocą regresji liniowej oraz grzbietowej, jak i modelami Prophet i \gls{mlp}. Według klasyfikacji Werona, regresja liniowa oraz regresja grzbietowa należą do zbiorów modeli statystycznych. Prophet również odpowiada kategorii modeli statystycznych, ale jego algorytm jest bardziej złożony od prostej regresji bądź szeroko używanych modeli ARIMA. Z kolei perceptron wielowarstwowy to przykład modelu computational intelligence, który wykorzystuje sieci neuronowe o różnej architekturze do prognozowania zmiennej docelowej.
