% !TEX program = xelatex
% !TeX encoding = utf8
% !TeX spellcheck = pl-PL

%%%%%%%%%%%%%%%%%%%%%%%%%%%%%%%%%%%%%%%%%%%%%%%%%%%%%%%%%%%%%%%%%%%%%%%%%%%
% Wybierz rodzaj pracy dyplomowej oraz wydział
% Pick thesis type and faculty
%%%%%%%%%%%%%%%%%%%%%%%%%%%%%%%%%%%%%%%%%%%%%%%%%%%%%%%%%%%%%%%%%%%%%%%%%%%
\documentclass[thesis=mgr,faculty=gik]{EE-dyplom} 

% thesis=[inz|mgr|bsc|msc]
%  * inz - praca inżynierska
%  * mgr - praca magisterska
%  * bsc - bachelor thesis
%  * msc - master thesis

% Skróty nazw wydziałów zgodne z domenami internetowymi
% Abbreviations of Faculties according to Internet subdomains
% faculty=[
%	arch,
%	gik,
%	ee,
%	wip
%	]

%%%%%%%%%%%%%%%%%%%%%%%%%%%%%%%%%%%%%%%%%%%%%%%%%%%%%%%%%%%%%%%%%%%%%%%%%%%
% Konfiguracja - do personalizacji
% Configuration - to be personalized
%%%%%%%%%%%%%%%%%%%%%%%%%%%%%%%%%%%%%%%%%%%%%%%%%%%%%%%%%%%%%%%%%%%%%%%%%%%
\instytut{Instytut Automatyki i Robotyki}
\kierunek{Automatyka, Robotyka i Informatyka Przemysłowa}
% \specjalnosc{Rzeczowidztwo}
\title{Modelowanie dynamiki cen na Rynku Dnia Następnego w Polsce}
\engtitle{Thesis name}
\album{335662}
\author{inż. Ivan Kaliankovich}
\promotor{prof. dr inż. Paweł Wnuk}
\date{2025}
\longdate{2025-05-01}

%\grantlicense{TRUE} % [TRUE|FALSE]

\streszczeniepracy{
Niniejsza praca poświęcona jest analizie i prognozowaniu cen energii elektrycznej na Rynku Dnia Następnego w Polsce w latach 2016-2023. Badanie skupia się na zrozumieniu dynamiki cen na tym rynku, który odgrywa kluczową rolę w zarządzaniu systemem elektroenergetycznym, umożliwiając elastyczne dostosowanie podaży do popytu w krótkim horyzoncie czasowym. Centralnym elementem pracy jest przygotowanie kompleksowej bazy danych, która obejmuje zarówno okresy stabilności cenowej, jak i momenty wysokiej zmienności, wynikające z czynników ekonomicznych, geopolitycznych oraz pogodowych. Uwzględniono szeroki zestaw danych, takich jak historyczne ceny energii, zapotrzebowanie, generacja energii z odnawialnych źródeł, ceny paliw, emisje CO$_2$, a także warunki atmosferyczne, co pozwoliło na holistyczne podejście do analizy.

Dobór zmiennych do analizy został oparty na gruntownym przeglądzie literatury, który wskazał na znaczenie takich czynników, jak opóźnione ceny energii, zapotrzebowanie systemowe, generacja energii odnawialnej oraz zmienne sezonowe i ekonomiczne, w tym efekty kalendarzowe i wahania cen surowców. Uwzględnienie tych aspektów umożliwiło dostosowanie badania do specyficznych uwarunkowań polskiego rynku energii, który charakteryzuje się unikalnymi wyzwaniami, takimi jak zależność od tradycyjnych źródeł energii i rosnąca rola odnawialnych źródeł energii. Analiza została przeprowadzona w dwóch różnych okresach, co pozwoliło na ocenę skuteczności stosowanych metod w zróżnicowanych warunkach rynkowych, od stabilnych po te o dużej niestabilności.

W badaniu wykorzystano zarówno modele statystyczne, jak i techniki uczenia maszynowego, dostosowując ich parametry, aby zwiększyć zdolność do przewidywania cen. Wyniki poddano ocenie za pomocą standardowych miar, analizując wpływ różnych zmiennych na jakość prognoz oraz porównując efektywność modeli w zależności od użytych danych. Praca rzuca światło na trudności związane z prognozowaniem w niestabilnych warunkach rynkowych, szczególnie w kontekście nagłych zmian cen wywołanych czynnikami zewnętrznymi, i wskazuje na konieczność dalszego rozwoju zaawansowanych metod analitycznych, takich jak podejścia hybrydowe, które lepiej radzą sobie z nieliniowością danych. Opracowanie może stanowić wsparcie dla uczestników rynku energii w podejmowaniu decyzji handlowych, zarządzaniu ryzykiem finansowym oraz planowaniu strategii w dynamicznie zmieniającym się środowisku energetycznym.
}

\slowakluczowe{Rynek Dnia Następnego, prognozowanie cen energii, zbiór danych}

\thesisabstract{
The following study focuses on the analysis and forecasting of electricity prices on the Day-Ahead Market in Poland over the period 2016-2023. The study aims to understand the price dynamics on this market, which plays a crucial role in managing the power system by enabling flexible adjustments of supply to demand in the short term. A central element of the research is the development of a comprehensive database covering both periods of price stability and high volatility, driven by economic, geopolitical, and weather-related factors. The dataset includes a wide range of variables such as historical electricity prices, demand, generation from renewable energy sources, fuel prices, CO$_2$ emissions, and weather conditions, facilitating a holistic approach to the analysis.

The selection of variables was guided by an in-depth literature review, highlighting the importance of factors such as lagged energy prices, system demand, renewable energy generation, and seasonal and economic variables, including calendar effects and raw material price fluctuations. Incorporating these aspects allowed the study to address the specific conditions of the Polish energy market, which faces unique challenges, such as reliance on traditional energy sources and the growing role of renewables. The analysis was conducted across two distinct periods, enabling an evaluation of model performance under varying market conditions, from stable to highly volatile.

The study employed both statistical models and machine learning techniques, adjusting their parameters to enhance predictive capabilities. The results were assessed using standard metrics, analyzing the impact of different variables on forecast accuracy and comparing model effectiveness based on the datasets used. The work sheds light on the challenges of forecasting in unstable market conditions, particularly in the context of sudden price changes triggered by external factors, and underscores the need for further development of advanced analytical methods, such as hybrid approaches, which better handle data nonlinearity. This research may support market participants in making trading decisions, managing financial risk, and planning strategies in a dynamically changing energy environment.
}

\thesiskeywords{Day-Ahead Market, electricity price forecasting, dataset}

\usepackage{float}
\makeglossaries

\begin{document}
    \frontpages % Strony nagłówkowe

    \chapter{Wstep}

\subsection*{Wprowadzenie}
\label{ch:wprowadzenie}
Rynek energii elektrycznej to jeden z filarów współczesnej gospodarki, a jego sprawne funkcjonowanie ma kluczowe znaczenie dla stabilności systemów elektroenergetycznych, przedsiębiorstw i codziennego życia konsumentów. W centrum rynku znajduje się \gls{rdn}, który działa jako platforma handlu energią na dzień przed jej dostarczeniem. RDN jest miejscem, gdzie producenci energii od elektrowni węglowych po farmy wiatrowe spotykają się z odbiorcami, takimi jak dostawcy energii dla domów mieszkalnych czy duże zakłady przemysłowe, w celu ustalenia ceny i wolumenów energii na każdą godzinę kolejnego dnia. Mechanizm działania RDN opiera się na systemie aukcyjnym: uczestnicy składają oferty kupna i sprzedaży, określając, ile energii są w stanie dostarczyć lub kupić oraz po jakiej cenie. System następnie dopasowuje te oferty, ustalając cenę równowagi, która obowiązuje dla danej godziny.

Taki model handlu pozwala na elastyczne reagowanie na zmieniające się warunki - zarówno po stronie podaży, jak i popytu. Na przykład, jeśli prognozy wskazują na silny wiatr, producenci energii wiatrowej mogą zwiększyć podaż, co może obniżyć ceny; z kolei fala upałów może zwiększyć zapotrzebowanie na energię do klimatyzacji, podnosząc ceny. RDN działa w wielu krajach, choć jego specyfika różni się w zależności od regionu. W Europie, w tym w Polsce, rynek ten jest częścią szerszego systemu integracji rynków energii, który ma na celu zapewnienie płynności i efektywności handlu. W Stanach Zjednoczonych RDN funkcjonuje w ramach regionalnych rynków, takich jak PJM Interconnection czy California ISO (CAISO), gdzie handel energią jest dodatkowo skomplikowany przez różnice regulacyjne między stanami. Niezależnie od regionu, RDN jest kluczowym narzędziem w zarządzaniu systemem elektroenergetycznym, umożliwiając szybkie dostosowanie podaży do popytu w krótkim horyzoncie czasowym.

Jednak handel na RDN to nie tylko szansa na zysk dla wszystkich biorących udział, ale i ogromne ryzyko finansowe, które wynika z nieprzewidywalności cen energii. Na rynku amerykańskim, gdzie mechanizm licytacji między kupującymi (buyers) a sprzedającymi (sellers) jest szczególnie rozwinięty, ryzyko to jest wyjątkowo widoczne. Uczestnicy rynku muszą składać oferty w czasie rzeczywistym, próbując przewidzieć, jak zachowa się cena w danej godzinie. Jeśli producent energii zaoferuje zbyt wysoką cenę, jego energia może nie zostać kupiona, co oznacza utratę przychodów; z kolei kupujący, który zaoferuje zbyt niską cenę, może zostać zmuszony do zakupu energii po znacznie wyższej cenie rynkowej. Przykładem skali tego ryzyka jest kryzys w Teksasie \cite{BUSBY2021102106} w lutym 2021 roku, znany jako Texas winter storm. W wyniku ekstremalnych mrozów i awarii systemu elektroenergetycznego ceny na rynku ERCOT (Electric Reliability Council of Texas) wzrosły do 9000 USD/MWh - poziomu, który dla wielu uczestników rynku oznaczał straty liczone w dziesiątkach milionów dolarów. W takich warunkach dokładna predykcja cen staje się nie tylko narzędziem do optymalizacji handlu, ale wręcz koniecznością, by uniknąć katastrofalnych strat. Na rynkach krajów rozwiniętych, gdzie \gls{oze} odgrywają coraz większą rolę, ceny mogą spadać do wartości ujemnych w godzinach nadprodukcji np. z energii słonecznej, by kilka godzin później gwałtownie wzrosnąć, gdy zapotrzebowanie przewyższa podaż. Ta zmienność sprawia, że handel na RDN przypomina grę o wysoką stawkę, w której każdy ruch musi być dokładnie przemyślany.

Rynek bilansujący stanowi kolejny istotny element systemu elektroenergetycznego, uzupełniając funkcjonowanie RDN. Działa on w czasie rzeczywistym, umożliwiając operatorom systemu elektroenergetycznego (w Polsce jest to Polskie Sieci Elektroenergetyczne, PSE) równoważenie podaży i popytu w sytuacjach, gdy rzeczywiste zużycie energii odbiega od prognoz ustalanych na RDN. Na rynku bilansującym uczestnicy mogą zgłaszać oferty na dostawy energii w bardzo krótkim horyzoncie czasowym, nawet w ciągu kilkunastu minut, co pozwala na szybkie reagowanie na nagłe zmiany, takie jak awarie elektrowni, nieoczekiwane skoki zapotrzebowania czy zmienność produkcji z OZE. Ceny na rynku bilansującym są często bardziej zmienne niż na RDN, co dodatkowo zwiększa ryzyko finansowe dla uczestników, ale jednocześnie podkreśla znaczenie precyzyjnych prognoz cen, które mogą pomóc w lepszym zarządzaniu tymi krótkoterminowymi wahaniami.

W niniejszej pracy magisterskiej skupiam się na analizie RDN w Polsce, który choć różni się od rynku amerykańskiego pod względem skali i regulacji, również zmaga się z podobnymi wyzwaniami. W Europie, w tym w Polsce, RDN jest częścią zintegrowanego systemu handlu energią, który opiera się na współpracy między krajami i dąży do harmonizacji rynków energii w ramach Unii Europejskiej. Mechanizm ustalania cen na RDN w Polsce opiera się na zasadzie jednolitej ceny, gdzie cena rozliczeniowa jest wyznaczana na podstawie równowagi popytu i podaży dla każdej godziny. Oznacza to, że wszyscy uczestnicy, których oferty zostały zaakceptowane, rozliczają się po tej samej cenie, co zapewnia przejrzystość i efektywność handlu. W Polsce RDN jest prowadzony przez Towarową Giełdę Energii (TGE), która od 2000 roku pełni rolę kluczowej platformy handlu energią. TGE organizuje aukcje na RDN, umożliwiając uczestnikom składanie ofert na każdą godzinę kolejnego dnia.W latach 2016-2023, które obejmują analizowany w pracy okres, ceny na RDN w Polsce wahały się od 200 PLN/MWh aż do ponad 3000 PLN/MWh, co odzwierciedla zarówno lokalne uwarunkowania (np. zależność od węgla, rosnąca rola OZE), jak i globalne trendy. Ważną rolą w takiej zmienności odegrały także niespodziewane czynniki zewnętrzne, zwane czarnymi łąbędziami. Z przykładów takich czynników można wymienić wybuch pandemii i wprowadzenie w związku z tym restrykcji na pracę oraz kryzys energetyczny w 2022 roku wywołany konfliktem zbrojnym na Ukrainie. Wszystkie te czynniki sprawiają, że prognozowanie cen energii elektrycznej na RDN w Polsce staje się niezwykle złożonym zadaniem, które wymaga uwzględnienia wielu zmiennych i zastosowania zaawansowanych metod analizy danych.

\subsection*{Cel pracy}
\label{ch:cel_pracy}

Większość dotychczasowych badań w literaturze naukowej dotycząca prognozowania cen energii elektrycznej, skupia się głównie na rynkach amerykańskich, europejskich oraz azjatyckich, pomijając specyfikę polskiego rynku energii. W związku z tym, celem niniejszej pracy jest opracowanie obszernej bazy danych z polskiego rynku energii, obejmującej zarówno okres umiarkowanej zmienności cen, jak i okres dużej zmienności cen i brak stabilności rynkowej. Baza danych zostanie przygotowana w celu uwzględnienia kluczowych czynników wpływających na ceny energii elektrycznej, charakterystycznych dla polskiego rynku. Następnie, w celu przetestowania użyteczności opracowanej bazy danych, przeprowadzono analizę z wykorzystaniem wybranych modeli uczenia maszynowego: regresji liniowej, regresji grzbietowej, modelu Prophet oraz wielowarstwowej sieci neuronowej.
    \makeatletter
    \@openrightfalse
    \makeatother
    \chapter{Przegląd literatury}
\label{ch:literatura}
Na podstawie ponadczasowej pracy prof. Rafała Werona poniższy rozdział omawia typowe podejścia w EPF, stosowane metody do predykcji cen oraz zmienne wejściowe. Badania prof. Werona związanego z Politechniką Wrocławską pasują do wszystkich rynków energii na świecie w tym rynku polskiego.

\section{Modele prognozowania}
\label{sec:modele_prognozowania_literatura}

Wśród stosowanych metod do predykcji cen energii elektrycznej wyróżnia się różne podejścia. Rafał Weron w swojej obszernej pracy z 2014 roku "Electricity price forecasting: A review of the state-of-the-art with a look into the future" \cite{WERON20141030} dokonuje przeglądu literatury z ubiegłych wtedy 15 lat, systematyzując szybko rosnącą liczbę publikacji w tej dziedzinie. Weron wyjaśnia mechanizmy kształtowania się cen na rynkach energii elektrycznej, koncentrując się na cenach dnia następnego. Klasyfikuje techniki predykcyjne pod względem horyzontu czasowego i zastosowanej metodologii. Wymienia następujące kategorie modeli: 
\begin{itemize}
    \item multi-agent
    \item fundamentalne,
    \item reduced-form,
    \item statystyczne,
    \item computational intelligence,
    \item hybrydowe.
\end{itemize}

Modele multi-agent symulują zachowanie uczestników rynku energii, takich jak producenci i konsumenci, w celu przewidywania cen. Weron \cite{WERON20141030} wskazuje, że modele oparte na równowadze Nasha-Cournota czy równowadze funkcji podaży, są przydatne w analizie długoterminowej, ale nie uwzględniają strategiczne obstawianie cen przez uczestników rynku. W związku z tym tego rodzaju metody pasują najbardziej do stabilnych rynków bez dużych wahań cenowych. 

Modele fundamentalne opierają się na analizie czynników ekonomicznych i fizycznych, takich jak ceny paliw, emisje CO2 czy zapotrzebowanie. Weron \cite{WERON20141030} dzieli je na parameter-rich (uwzględniające wiele zmiennych) i parsimonious structural (uproszczone). Jednym z wyzwań takich modeli jest ich złożoność i wymóg dużej ilości danych, które często mogą być niedostępne w czasie rzeczywistym. 

Modele reduced-form opisują dynamikę cen za pomocą procesów stochastycznych, takich jak jump-diffusions czy Markov regime-switching. Weron \cite{WERON20141030} wskazuje, że są one użyteczne w modelowaniu dziennej zmienności cen, ale mogą nie być dokładne w próbie dokładnego liczenia cen godzinowych, gdyż nie uwględniają wpływu zmiennych fundamentalnych, takich jak sezonowość czy zmiany w podaży i popycie.

Modele statystyczne, takie jak ARIMA i GARCH, są szeroko stosowane w krótkoterminowym EPF. Modele autoregresyjne (AR) wykorzystują liniową kombinację przeszłych wartości zmiennej do prognozowania przyszłych wartości. Modele średniej ruchomej (MA) prognozują zmienną na podstawie liniowej kombinacji przeszłych błędów prognoz. Modele ARMA łączy te dwa podejścia, a modele ARIMA dodają różnicowanie, aby uwzględnić niestacjonarność danych. W celu uwzględnienia sezonowości wykorzystuje się również model SARIMA. Zgodnie z \cite{appliedmath3020018} modele SARIMA wykazują dobre wyniki w prognozowaniu cen energii elektrycznej, ale ich skuteczność może być ograniczona w przypadku danych o dużej zmienności. Weron \cite{WERON20141030} wspomina również o modelach ARX i ARMAX, gdzie X odpowiada za zmienne zewnętrzne objaśniające. Model GARCH jest szczególnie przydatny w modelowaniu zmienności cen energii elektrycznej, ponieważ uwzględnia heteroskedastyczność i zmienność w czasie.

Modele computational intelligence, oparte na technikach uczenia maszynowego, zyskały na popularności w nowszych badaniach. Wśród popularnych metod wyróżniają się metody rozmyte, metody wektorów nośnych, LSTM oraz CNN. Mikołaj Kalisz i Adam Mantiuk z Politechniki Warszawskiej \cite{MGR2025} używają Perceptron wielowarstwowego (MLP). W pracy Grzegorza Marcjasza \cite{en13184605} proponowany jest dobór hiperparametrów do głębokiej sieci neuronowej (DNN). Badanie porównuje DNN do statystycznego modelu LASSO. W tym badaniu pokazuje się, że wyniki DNN są lepsze od wybranego modelu statystycznego, co może wskazywać na przewagę uczenia maszynowego nad tradycyjnymi metodami statystycznymi.

Modele hybrydowe łączą elementy różnych kategorii, aby wykorzystać ich zalety. Jinliang Zhang jest przykładem takiego podejścia. W pracy \cite{TAN20103606} łączy transformację falkową z ARIMA i GARCH. W tej pracy argumentuje się, że połączenie WT z modelami ARIMA i GARCH pozwala na skuteczne modelowanie złożonych cech cen energii elektrycznej, takich jak niestacjonarność, nieliniowość i wysoka zmienność. W późniejszej pracy Zhang \cite{ZHANG2012695} łączy wspomniane transformację falkową oraz ARIMA z metodę najmniejszych kwadratów maszyn wektorów nośnych (LSSVM). Potwierdzając skuteczność takich metód, Weron \cite{WERON20141030} podaje, że popularnym podejściem hybrydowym jest łączenie modeli statystycznych z sieciami neuronowymi, co pozwala na modelowanie zarówno liniowych, jak i nieliniowych zależności.

W niniejszej pracy planuje się zastosowanie modeli Prophet i MLP do prognozowania cen energii elektrycznej na RDN. Prophet najbardziej pasuje do kategorii statystycznych, ale jego algorytm jest bardziej złożony niż tradycyjne modele ARIMA. MLP to przykład modelu computational intelligence, który wykorzystuje sieci neuronowe do prognozowania cen energii elektrycznej.

\section{Zmienne wejściowe}
\label{sec:zmienne_wejsciowe_literatura}






    \chapter{Dane i zmienne}
\label{ch:dane}

\section{Opis zbioru danych}
Dane odgrywają kluczową rolę w analizie i prognozowaniu cen energii na Rynku Dnia Następnego (RDN) w Polsce, stanowiąc fundament dla modeli uczenia maszynowego zastosowanych w niniejszej pracy. Zbiór danych wykorzystany w badaniu obejmuje okres od 1 stycznia 2016 roku do 31 grudnia 2024 roku i zawiera dane godzinowe, co powinno było dać łącznie 78 912 rekordów. Niestety niektóre dane są niekompletne i zawiera braki lub NaN wartości. Na szczęście, takich braków było stosunkowo nie dużo do kompletnych danych i postanowiłem pozbyć się rekordów o pustych parametrach. W efekcie, po usunięciu niekompletnych danych, pozostało 78 814 rekordów, co stanowi solidną podstawę do analizy i modelowania. Zbiór danych zawiera różnorodne zmienne, które można podzielić na kilka kategorii, które zostaną opisane w tym dziale. Większość danych pochodzi z \gls{pse} \cite{PSEOLD}, czyli otwartej platformie, która dostarcza różnego rodzaju danych dostępnych dla analizy. Warto zauważyć, że od 14 czerwca 2024 roku PSE zmieniła sposób raportowania danych i przeszła na nową stronę \cite{PSENEW}, w związku z tym PSE posiada dwie oddzielne strony do raportów historycznych przed i po tym dniu. Wiele z danych zostały pobrane również ze strony energy.instrat.pl. Jest to strona, która pobiera dane z platformy PSE i udostępnia w sposób wygodniejszy, ponieważ dane są do pobrania jednym klikiem za dowolny okres czasu z dowolną częstotliwością. Dane dotyczące cen paliw kopalnych oraz kursów walut (np. PLN/USD) zostały pozyskane z innych źródeł, takich jak publiczne bazy danych rynkowych i platformy finansowe. Niniejszy rozdział szczegółowo opisuje zmienną zależną i zmienne niezależne, podzielone na kategorie, a także prezentuje kluczowe cechy danych za pomocą tabel i wykresów, co pozwala na lepsze zrozumienie ich specyfiki i wyzwań związanych z modelowaniem.

\section{Zmienna zależna}
Zmienna zależna w niniejszej pracy to fixing\_i\_price, czyli cena energii elektrycznej na Rynku Dnia Następnego (RDN) w Polsce, wyrażona w PLN/MWh. Dane dotyczące tej zmiennej zostały pobrane z wymienionej platformy energy.instrat.pl w granulacji godzinowej. Zbiór danych obejmuje okres od 1 stycznia 2016 roku do 31 grudnia 2024 roku. Statystyki opisowe zmiennej fixing\_i\_price przedstawiono w Tabeli poniżej. 
\begin{table}[H]
    \centering
    \begin{tabular}{|l|r|}
    \hline
    \textbf{Statystyka} & \textbf{Wartość} \\ \hline
    Średnia             & 344.10 PLN/MWh   \\ \hline
    Odchylenie std.     & 268.35 PLN/MWh   \\ \hline
    Minimum             & -360.00 PLN/MWh  \\ \hline
    25\% (Q1)           & 176.11 PLN/MWh   \\ \hline
    Mediana             & 250.00 PLN/MWh   \\ \hline
    75\% (Q3)           & 434.84 PLN/MWh   \\ \hline
    Maksimum            & 3812.45 PLN/MWh  \\ \hline
    \end{tabular}
    \caption{Podstawowe statystyki zmiennej fixing\_i\_price}
    \label{tab:fixing-i-price-stats}
\end{table}

Aby lepiej zrozumieć dynamikę cen energii na RDN, przeanalizowano ich zmienność w całym okresie badania. Rysunek poniżej przedstawia zmienność cen energii w czasie zebranych danych.

\begin{figure}[H]
    \centering
    \includegraphics[width=\textwidth]{../plots/data/quarterly_fixing_i_price.png}
    \caption{Zmienność cen energii elektrycznej na RDN w latach 2016–2024}
    \label{fig:fixing-i-price-trend}
\end{figure}

Widać wyraźne różnice w poziomie cen w różnych okresach: od 2016 do Q4 roku 2020 ceny były stosunkowo stabilne, oscylując w przedziale 100–300 PLN/MWh. Sytuacja zmieniła się w 2020 roku, gdy zaczęły pojawiać się pierwsze skoki cenowe z powodu poważnych obostrzeń z powodu pandemii, a w 2022 roku, w wyniku kryzysu energetycznego wywołanego wojną na Ukrainie i ograniczeniami w dostawach paliw kopalnych, ceny osiągnęły rekordowe poziomy. Pierwszy okres zostanie określony jako okres stabilności cenowej, a drugi jako okres skoków cenowych. Te dwa okresy pokazują, jak niekorzystne sytuacje gospodarcze mogą wpływać na dynamikę cen energii, co ma istotne implikacje dla modelowania i prognozowania.

\begin{figure}
    \centering
    \includegraphics[width=\textwidth]{../plots/data/histogram_fixing_i_price.png}
    \caption{Histogram rozkładu zmiennej fixing\_i\_price}
    \label{fig:fixing-i-price-histogram}
\end{figure}

Rysunek \ref{fig:fixing-i-price-histogram} przedstawia histogram rozkładu zmiennej \texttt{fixing\_i\_price}. Rozkład jest wyraźnie asymetryczny, z długim prawym ogonem, co odzwierciedla występowanie skoków cenowych, takich jak te w 2022 roku. Ujemne ceny, choć rzadkie (ok. 0,4\% rekordów), są widoczne w lewej części histogramu, co potwierdza specyficzne cechy danych i potrzebę stosowania odpowiednich metod modelowania.

\section{Zbiór zmiennych niezależnych}
Dobór zmiennych niezależnych jest bardzo ważny dla osiągnięcia dobrych wyników badania. W niniejszej pracy wykorzystano różnorodne zmienne niezależne, które można podzielić na kilka kategorii. Obejmują one dane pogodowe, zapotrzebowanie, straty sieciowe, bilanse wymiany transgranicznej, dane o produkcji energii przez poszczególne typy generatorów, ceny paliw kopalnych, emisji CO$_2$ i inne. Wybór tych zmiennych oparty jest na ich potencjalnym wpływie na ceny energii elektrycznej. Poniżej przedstawiono szczegółowy opis każdej z kategorii zmiennych niezależnych, które zostały uwzględnione w analizie.

\subsection{Dane pogodowe}
Pierwotnie zbiór danych miał być zestawiony z danych dostępnych za pomocą oficjalnej strony Instytutu Meteorologii i Gospodarki Wodnej, natomiat dane historyczne z lat 2016-2024 mają ograniczoną rozdzielczość. Zbierane są przez wiele stacji meteorologicznych, które są rozproszone po całym kraju, ale tylko w godzinach 6:00, 12:00 oraz 18:00. Aproksymować dane pogodowe w godzinach nocnych jest zadaniem nie do wykonania, szczególnie w przypadku sezonów zimowych, gdzie temperatura w nocy może drastycznie spadać w ciągu godziny. Z tego powodu jako źródło danych pogodowych wykorzystano stronę open-meteo.com \cite{METEO}. Jest to strona, która zbiera dane z różnych stacji meteorologicznych i udostępnia je w formie API. Dzięki temu można pobrać dane pogodowe dla dowolnego okresu czasu i lokalizacji.

Dane pogodowe zostały pobrane dla czterech lokalizacji w Polsce: Warszawy (WAW), Koszalina (KSZ), Krakowa (KRK) i Babimost (BAB), a następnie dopasowane do godzinowego formatu danych RDN, co pozwoliło na ich integrację z pozostałymi zmiennymi. Wybór miast został podyktowany ich zróżnicowaniem geograficznym i klimatycznym, co pozwala uwzględnić regionalne różnice w warunkach pogodowych wpływających na produkcję i zapotrzebowanie na energię. Warszawa, jako stolica i największe miasto Polski, reprezentuje centralny region kraju o wysokim zapotrzebowaniu na energię, szczególnie w okresach zimowych i letnich. Koszalin, położony na Pomorzu, jest kluczowy ze względu na bliskość farm wiatrowych na Morzu Bałtyckim, co czyni go istotnym punktem dla analizy produkcji energii wiatrowej. Kraków, znajdujący się w południowej Polsce, charakteryzuje się większym udziałem energii słonecznej w miksie energetycznym, a także wysokim zapotrzebowaniem na energię w sezonie grzewczym z powodu zanieczyszczenia powietrza i częstego stosowania ogrzewania elektrycznego. Babimost, zlokalizowany w zachodniej Polsce, jest istotny ze względu na swoje położenie w pobliżu granicy z Niemcami.

Parametry pogodowe zostały wybrane z uwzględnieniem ich bezpośredniego wpływu na rynek energii. Temperatura jest kluczowym czynnikiem, ponieważ wpływa na zapotrzebowanie na energię – niskie temperatury zwiększają zużycie energii na ogrzewanie, natomiast wysokie temperatury latem podnoszą zapotrzebowanie na klimatyzację. Prędkość wiatru mierzona na wysokości 100 metrów nad powierzchnią ziemi została wybrana, ponieważ jest to przeciętna wysokość dla turbin wiatrowych w Polsce, co pozwala dokładniej oszacować potencjalną produkcję energii z farm wiatrowych. Promieniowanie słoneczne jest istotne dla produkcji energii z paneli fotowoltaicznych. Zachmurzenie zostało uwzględnione, ponieważ wysoki poziom zachmurzenia zmniejsza efektywność paneli słonecznych, co może zwiększać ceny energii poprzez ograniczenie podaży z OZE. Wybór tych parametrów pozwala na kompleksową analizę wpływu pogody na ceny energii na RDN.

Poniżej przedstawię wykresy dla każdego z parametrów pogodowych, które zostały uwzględnione w analizie. Wykresy przedstawiają zmienność danych pogodowych w czasie. Zachmurzenie jest wyrażone w oktantach (0-8), gdzie 0 oznacza brak zachmurzenia, a 8 oznacza całkowite zachmurzenie.

\begin{figure}[H]
    \centering
    \includegraphics[width=\textwidth]{../plots/data/temp_time_series_full.png}
    \caption{Zmienność temperatury w czasie (2016–2024)}
    \label{fig:temp-time-series-full}
\end{figure}

\begin{figure}[H]
    \centering
    \includegraphics[width=\textwidth]{../plots/data/wind_speed_time_series_full.png}
    \caption{Zmienność prędkości wiatru w czasie (2016–2024)}
    \label{fig:wind-speed-time-series-full}
\end{figure}

\begin{figure}[H]
    \centering
    \includegraphics[width=\textwidth]{../plots/data/solar_radiation_time_series_full.png}
    \caption{Zmienność promieniowania słonecznego w czasie (2016–2024)}
    \label{fig:solar-radiation-time-series-full}
\end{figure}

\begin{figure}[H]
    \centering
    \includegraphics[width=\textwidth]{../plots/data/cloud_cover_time_series_full.png}
    \caption{Zmienność zachmurzenia w czasie (2016–2024)}
    \label{fig:cloud-cover-time-series-full}
\end{figure}

Każdy z wykresów przedstawia zmienność danego parametru pogodowego w wybranych lokalizacjach w przeciągu okresu badawczego. Wyraźnie widać sezonowe wahania parametrów pogodowych, co jest typowe dla klimatu Polski. Temperatura i promieniowanie słoneczne mają wyraźnie większe wartości w sezonach letnich, prędkość wiatru w sezonach zimowych, a zachmurzenie ma bardziej zróżnicowany charakter.

Chciałbym również przybliżyć wykresy i potwierdzić efekt sezonowości na przykładzie 2022 roku. 

\begin{figure}[H]
    \centering
    \includegraphics[width=\textwidth]{../plots/data/temp_time_series_2022.png}
    \caption{Zmienność temperatury w czasie (2022)}
    \label{fig:temp-time-series-2022}
\end{figure}

\begin{figure}
    \centering
    \includegraphics[width=\textwidth]{../plots/data/wind_speed_time_series_2022.png}
    \caption{Zmienność prędkości wiatru w czasie (2022)}
    \label{fig:wind-speed-time-series-2022}
\end{figure}

\begin{figure}
    \centering
    \includegraphics[width=\textwidth]{../plots/data/solar_radiation_time_series_2022.png}
    \caption{Zmienność promieniowania słonecznego w czasie (2022)}
    \label{fig:solar-radiation-time-series-2022}
\end{figure}

\begin{figure}
    \centering
    \includegraphics[width=\textwidth]{../plots/data/cloud_cover_time_series_2022.png}
    \caption{Zmienność zachmurzenia w czasie (2022)}
    \label{fig:cloud-cover-time-series-2022}
\end{figure}

W ramach analizy w późniejszej części pracy spróbuję uprościć model i uśrednić wartości parametrów pogodowych dla całego kraju i sprawdzić czy wynik się poprawi. 

\subsection{Produkcja energii z wybranych źródeł}
Zmienne dotyczące produkcji energii z różnych źródeł odgrywają kluczową rolę w analizie cen energii na Rynku Dnia Następnego (RDN), ponieważ odzwierciedlają strukturę podaży energii w Polsce, która ma bezpośredni wpływ na dynamikę cen. W niniejszej pracy uwzględniono osiem zmiennych opisujących produkcję energii: 
\begin{itemize}
    \item \textbf{hard\_coal} - produkcja z węgla kamiennego (MW),
    \item \textbf{coal\_derived} - produkcja z paliw pochodnych węgla (MW),
    \item \textbf{lignite} - produkcja z węgla brunatnego (MW),
    \item \textbf{gas} - produkcja z gazu ziemnego (MW),
    \item \textbf{oil} - produkcja z ropy naftowej lub jej pochodnych (MW),
    \item \textbf{biomass} - produkcja z biomasy (MW),
    \item \textbf{wind} - produkcja z elektrowni wiatrowych lądowych (MW),
    \item \textbf{solar} - produkcja z paneli fotowoltaicznych (MW).
\end{itemize}

Dane te pochodzą z Polskich Sieci Elektroenergetycznych (PSE) i zostały dopasowane do godzinowego formatu danych RDN, co pozwoliło na ich integrację z pozostałymi zmiennymi. 
\begin{figure}[H]
    \centering
    \includegraphics[width=1.1\textwidth]{../plots/data/energy_production_time_series_full.png}
    \caption{Zmienność produkcji energii z różnych źródeł w czasie (2016–2024)}
    \label{fig:energy-production-time-series-full}
\end{figure}

Wykres powyżej przedstawia średnią miesięczną produkcję energii z różnych źródeł w Polsce w latach 2016–2024, co jest istotne w kontekście prognozowania cen energii (EPF) na Rynku Dnia Następnego (RDN). Wykres obszarowy ukazuje dominację węgla kamiennego i brunatnego, której do tej pory odpowiadają za większość produkcji energii elektrycznej. Biomasa oraz paliwa pochodne węgla stanowią znikomą część podaży i mogą być pominięte dla redukcji ilości zmiennych. Produkcja za pomocą gazu i ropy stale posiada niewielką, ale istotną część produkcji energii. Warto zauważyć, że produkcja z OZE, szczególnie ze słońca znacząco rośnie w ostatnich latach, co może mieć istotny wpływ na ceny energii i umiejętność jej prognozowania. W najbardziej korzystne dla gospodatki momenty część produkcji z OZE może przekraczać zapewnienie energii z węgla, co może prowadzić do spadku cen energii. Warto również zauważyć, że produkcja z węgla kamiennego i brunatnego jest bardziej stabilna i przewidywalna niż produkcja z OZE, co może wpływać na dokładność prognoz. W związku z tym, zmienne dotyczące produkcji energii z różnych źródeł są istotnym elementem analizy i modelowania cen energii na RDN.

\subsection{Handel energią z państwami sąsiednimi}
Zmienne dotyczące wymiany energii z innymi krajami są istotnym elementem analizy cen energii na Rynku Dnia Następnego (RDN), ponieważ pozwalają na uwzględnienie wpływu handlu międzynarodowego na ceny energii w Polsce. W niniejszej pracy uwzględniono następujące zmienne opisujące wymianę energii z innymi krajami:

    \chapter{Eksploracja danych}
\label{sec:eksploracja}

W niniejszym rozdziale przedstawiono proces przygotowania wyżej omówionego zestawu danych. Proces eksploracji danych obejmował obsługę braków wartości, dostosowanie granulacji zmiennych, analizę korelacji, podział danych na okresy o różnej zmienności, podział na zbiory treningowy, walidacyjny i testowy oraz preprocessing danych przed modelowaniem.

\section{Wstępna obróbka danych}
\subsubsection{Obsługa braków wartości}
Pierwszym krokiem w eksploracji danych była analiza i obsługa braków wartości w zmiennych, takich jak cena energii, zapotrzebowanie, czy bilanse handlowe. Braki wartości w zmiennych objaśniających mogą wynikać z przyczyny, że w trakcie niektórych godzin pomiary nie były zbierane, lub z powodu błędów w danych, które mogły prowadzić do ich usunięcia. Na szczęście pierwotny zbiór danych zawierał niewiele braków i były one losowe w kontekście całego zbioru. Z oczekiwanych 78912 rekordów godzinowych pozostało 78451. Jest to zaledwie 0.55\% braków, co jest nie powinno stanowić problemu w dalszej analizie.

Niektóre zmienne, takie jak kursy walut czy ceny paliw kopalnych miały braki w okresach zamkniętego rynku, czyli weeekendy i święta. W takich przypadkach brakujące wartości zostały przeniesione z poprzedniego dnia handlu. 

\subsubsection{Obsługa granulacji}
Dane użyte w pracy charakteryzowały się różną granulacją czasową. PSE udostępnia dane dotyczące sieci energii elektrycznej, w tym handlu w rozdzielczości godzinowej. Zmienne pogodowe również mają granulację dzienną. Natomiast zmienne makroekonomiczne, takie jak ceny paliw, są dostępne w częstotliwości dniowej lub tygodniowej.

Aby ujednolicić granulację do poziomu godzinowego, zastosowano dwie techniki. W przypadku zmiennych o granulacji dziennej, takich jak dane pogodowe, założono, że wartości w ciągu doby nie ulegają zmianie. Wartości dzienne przypisano więc każdej godzinie danego dnia, co pozwoliło na zachowanie prostoty przy jednoczesnym dostosowaniu danych do godzinowej rozdzielczości ceny energii. 

Dla zmiennych o rzadszych granulacji, na przykład dla ceny węgla (miesięczna) oraz emisji CO2 (tygodniowa), przeprowadzono uzupełnianie danych metodą interpolacji liniowej między sąsiednimi wartościami. Poniższy przykład dla granulacji tygodniowej.
\[
\text{cena w dniu } d = \text{cena w tygodniu } t + \left( \frac{\text{cena w tygodniu } t+1 - \text{cena w tygodniu } t}{6} \right) \times d,
\]
Następnie, podobnie jak w przypadku danych dziennych, wartości te przypisano każdej godzinie w danej dobie. Podejście to umożliwiło ujednolicenie wszystkich zmiennych do godzinowej rozdzielczości danych docelowych, co było niezbędne dla weryfikacji zbioru danych.

\section{Analiza korelacji}

Aby zbadać zależności między zmiennymi objaśniającymi a zmienną docelową, przeprowadzono analizę korelacji, wykorzystując dwa współczynniki: Pearsona i Spearmana. Wybór odpowiedniego współczynnika dla każdej zmiennej oparto na charakterze jej relacji z \texttt{fixing\_i\_price}, co pozwoliło na bardziej precyzyjne oszacowanie siły i rodzaju zależności.

Współczynnik korelacji Pearsona (\( r \)) mierzy liniową zależność między dwiema zmiennymi. Jest on zdefiniowany wzorem:
\[
r = \frac{\sum_{i=1}^{n} (x_i - \bar{x})(y_i - \bar{y})}{\sqrt{\sum_{i=1}^{n} (x_i - \bar{x})^2} \sqrt{\sum_{i=1}^{n} (y_i - \bar{y})^2}},
\]
gdzie \( x_i \) i \( y_i \) to wartości zmiennych i danej zmiennej objaśniającej), \( \bar{x} \) i \( \bar{y} \) to ich średnie, a \( n \) to liczba obserwacji. Współczynnik Pearsona przyjmuje wartości w przedziale \([-1, 1]\), gdzie \( r = 1 \) oznacza doskonałą dodatnią zależność liniową, \( r = -1 \) doskonałą ujemną zależność liniową, a \( r = 0 \) brak liniowej zależności.

Pearson jest odpowiedni dla zmiennych, których relacja jest liniowa. Na przykład wzrost zapotrzebowania zwykle większa ceny energii liniowo.

Współczynnik korelacji Spearmana (\( \rho \)) mierzy monotoniczną zależność między zmiennymi, co czyni go bardziej odpowiednim dla relacji nieliniowych. Spearman opiera się na rangach wartości zmiennych, a jego wzór to:

\[
\rho = 1 - \frac{6 \sum_{i=1}^{n} d_i^2}{n(n^2 - 1)},
\]

gdzie \( d_i \) to różnica między rangami wartości \( x_i \) i \( y_i \). Współczynnik Spearmana również przyjmuje wartości w przedziale \([-1, 1]\), ale nie zakłada liniowości relacji – wystarczy, że wzrost jednej zmiennej odpowiada wzrostowi (lub spadkowi) drugiej w sposób monotoniczny.

Spearman jest szczególnie użyteczny dla zmiennych o nieliniowej relacji z \texttt{fixing\_i\_price}, takich jak \texttt{co2\_price}, \texttt{coal\_pscmi1\_pln\_per\_gj} czy \texttt{solar}. Na przykład wzrost cen CO2 w 2022 roku prowadził do nieproporcjonalnego wzrostu cen energii, co lepiej oddaje Spearman niż Pearson.

W celu wyboru odpowiedniego współczynnika korelacji obliczono zarówno korelację Pearsona, jak i Spearmana dla wszystkich zmiennych objaśniających względem \texttt{fixing\_i\_price}. Następnie obliczono bezwzględną różnicę między tymi współczynnikami (\(| \rho - r |\)). Zmienne, dla których różnica była większa niż 0,1, uznano za posiadające nieliniową relację z \texttt{fixing\_i\_price}, stosując dla nich korelację Spearmana. W pozostałych przypadkach wybrano korelację Pearsona, zakładając liniową zależność. 

Analiza wykazała, że zmienne takie jak \texttt{co2\_price} (różnica 0,114), \texttt{coal\_pscmi1\_pln\_per\_gj} (0,161), \texttt{pln\_usd} (0,112), \texttt{solar} (0,172), \texttt{gas} (0,153), \texttt{biomass} (0,174), \texttt{coal-derived} (0,142) oraz zmienne związane z promieniowaniem słonecznym (różnice 0,106-0,115) mają nieliniową relację z \texttt{fixing\_i\_price}. Nieliniowość wynika z charakteru tych zmiennych – na przykład wysoka produkcja energii z fotowoltaiki w miesiącach letnich obniża ceny energii w sposób nieproporcjonalny, co prowadzi do bardzo niskich, a czasem ujemnych cen. Podobnie gwałtowny wzrost cen CO2 w okresie niespokojnym (2020-2023) zwiększał koszty produkcji energii w elektrowniach węglowych, ale efekt ten był wzmacniany przez inne czynniki, takie jak spekulacje rynkowe.

W wyniku tego powstał następujący wykres korelacji poniżej

\begin{figure}[H]
    \centering
    \includegraphics[width=0.9\textwidth]{../plots/correlation_with_fixing_i_price.png}
    \caption{Wykres korelacji zmiennych objaśniających względem zmiennej docelowej \texttt{fixing\_i\_price}.}
    \label{fig:correlation_plot}
\end{figure}

Zgodnie z literaturą największy wpływ na ceny energii mają zmienne, które powstały w wyniku przetworzenia zmiennej objaśnianej, czyli ceny opóźnione i średnie. Koljenymi zmiennymi wykazującymi silną korelację są ceny na surowce, kurs pollskiego złotego oraz zapotrzebowanie i wolumen. Sprzeczna z logiką może być pozytywna korelacja ceny z zmienną \texttt{solar}, która wskazuje na produkcję energii z paneli fotowoltaicznych. Wynika to z faktu, w momentach, gdy świeci słońce i produkcja energii z OZE jest wysoka, zapotrzebowanie również jest zwiększone i to powoduje wzrost cen. Z kolei wartości reprezentujące parametry pogodowe nie wykazują istotnej korelacji. 

Produkcja energii z odnawialnych źródeł i gazu również odgrywa rolę. Zmienna \texttt{solar} (Spearman: 0,490) wskazuje, że wysoka produkcja energii z fotowoltaiki obniża ceny energii, szczególnie w miesiącach letnich, gdzie nadpodaż energii z OZE może prowadzić do bardzo niskich cen. \texttt{gas} (Spearman: 0,449) pokazuje, że produkcja energii z gazu ma nieliniowy wpływ, zależny od cen gazu i dostępności innych źródeł energii. Ponadto zmienne takie jak \texttt{fixing\_i\_volume} (Spearman: 0,442), \texttt{power\_loss} (Spearman: 0,441), \texttt{oil} (Spearman: 0,349) oraz \texttt{Load} (Pearson: 0,256) mają umiarkowany wpływ, odzwierciedlając znaczenie wolumenu obrotu, strat w sieci, produkcji z oleju oraz zapotrzebowania na energię.

\subsubsection{Zbiór danych skrócony}
Na podstawie analizy korelacji stworzony został skrócony zbiór danych, który z pierwotnych 54 regresorów zostawia najbardziej istotne. Tymi zmiennymi zostały wszystkie przekraczające próg istotności na poziomie \texttt{0.2},  zmienne sezonowe określające dzień tygodnia, miesiąc, godzinę oraz święta, średnia arytmetyczna ze zmiennych objaśniających prędkość wiatru w Polsce, gdyż wykazują one najmocniejszą odwrotną korelację oraz zmienna objaśniająca generację OZE w procentach, ponieważ jest to ważna zmienna z punktu widzenia literatury. W wyniku tego powstał zbiór danych z 27 zmiennymi objaśniającymi. Taki zbiór danych został określony jako \texttt{zbiór danych skrócony} i będzie użyty w dalszej części pracy do analizy w celu sprawdzenia istotności zbioru danych o największej ilości parametrów. Poniżej przedstawiono mapę cieplną dla zmiennych objaśniających w zbiorze danych skróconym bez uwzględnienia zmiennych opóźnioncyh i średnich. 

\begin{figure}[H]
    \centering
    \includegraphics[width=0.9\textwidth]{../plots/heatmap_short_db_features.png}
    \caption{Mapa cieplna korelacji zmiennych objaśniających w skróconym zbiorze danych.}
    \label{fig:heatmap_shortened_dataset}
\end{figure}

Z powyższej mapy cieplnej wynika, że zmienne o dużej korelacji z ceną energii w dużym stopniu są skorelowane między sobą. 

\section{Podział danych}

\subsubsection{Podział na okresy spokojny i niespokojny}

Dane zostały podzielone na dwa okresy w celu uwzględnienia różnych warunków rynkowych i ich wpływu na ceny energii. Okres spokojny (2016-2019) charakteryzuje się stabilnymi cenami energii, wynikającymi z braku znaczących szoków podażowych, przewidywalnych cen paliw oraz stopniowego wzrostu cen CO2 w ramach polityki klimatycznej UE. W tym okresie nie występowały większe kryzysy geopolityczne ani pandemie, co pozwoliło na utrzymanie cen w stosunkowo wąskim zakresie.

Okres niespokojny (2020-2023) został zdominowany przez szereg wydarzeń, które drastycznie wpłynęły na rynek energii. Pandemia COVID-19 w latach 2020-2021 początkowo obniżyła zapotrzebowanie na energię, ale ożywienie gospodarcze w 2021 roku spowodowało gwałtowny wzrost cen. Kryzys energetyczny w latach 2021-2022, związany z ograniczoną podażą gazu, rekordowymi cenami CO2 oraz wysokimi cenami węgla, doprowadził do ekstremalnych skoków cen energii. Wybuch wojny na Ukrainie w 2022 roku dodatkowo zaostrzył sytuację, powodując przerwanie dostaw gazu z Rosji, sankcje i spekulacje rynkowe, co przełożyło się na rekordowe ceny energii. W tym okresie pojawiły się również ujemne ceny, wynikające z nadpodaży energii z OZE i ograniczonej elastyczności systemu elektroenergetycznego.

Statystyki opisowe dla obu okresów przedstawiono w tabeli~\ref{tab:periods_stats_comparison}. Okres spokojny charakteryzuje się niższą średnią ceną, mniejszą zmiennością i dużo mniejszym odchyleniem standardowym, co odzwierciedla stosunkowo stabilne warunki rynkowe. W okresie niespokojnym średnia cena wzrosła, a współczynnik zmienności i odchylenie znacząco rosną. W latach 2020-2023 pojawiły się również ujemne ceny oraz rekordowe maksima. Podział na te dwa okresy i poddanie ich osobnej analizie pozwoli lepiej ocenić skuteczność wybranych cech i modeli w różnych warunkach rynkowych.

\begin{table}[h]
    \centering
    \caption{Porównanie statystyk opisowych cen energii w okresach spokojnym (2016-2019) i niespokojnym (2020-2023).}
    \label{tab:periods_stats_comparison}
    \begin{tabular}{|l|c|c|}
        \hline
        \textbf{Miara} & \textbf{Okres spokojny} & \textbf{Okres niespokojny} \\
        \hline
        Średnia (PLN/MWh) & 193,51 & 478,06 \\
        \hline
        Mediana (PLN/MWh) & 182,00 & 412,00 \\
        \hline
        Odchylenie standardowe (PLN/MWh) & 70,67 & 321,39 \\
        \hline
        Współczynnik zmienności (\%) & 36,52 & 67,23 \\
        \hline
        Kwartyl Q1 (25\%) (PLN/MWh) & 143,66 & 246,41 \\
        \hline
        Kwartyl Q3 (75\%) (PLN/MWh) & 229,35 & 609,00 \\
        \hline
        Minimum (PLN/MWh) & 31,00 & -50,00 \\
        \hline
        Maksimum (PLN/MWh) & 1199,53 & 3812,45 \\
        \hline
        Procent dni z ceną powyżej 500 PLN/MWh (\%) & 0,41 & 36,94 \\
        \hline
    \end{tabular}
\end{table}

\subsubsection{Podział danych na zbiory treningowe i testowe}

Dane zostały podzielone na zbiory treningowe i testowe w obrębie każdego z dwóch okresów, aby uwzględnić strukturę szeregów czasowych. Podział został przeprowadzony sekwencyjnie w proporcji 75/25, co zapewnia dużą ilość danych do treningu (3 lata) oraz odpowiednią ilość danych do testowania (1 rok). W przypadku takiego podziału w zbiorach testowych można przetestować wszystkie okresy sezonowe.

\begin{figure}[h]
    \centering
    \includegraphics[width=0.9\textwidth]{../../plots/periods_split_combined.png}
    \caption{Podział szeregów czasowych cen energii na zbiory treningowe i testowe}
    \label{fig:periods_split_combined}
\end{figure}

\section{Przygotowanie danych}

Przed przystąpieniem do modelowania dane zostały poddane szeregu kroków preprocessingu, aby zapewnić ich odpowiednią jakość i format dla wybranych modeli.

Pierwszym krokiem jest kodowanie zmiennych cyklicznych. Zmienne sezonowe mają charakter cykliczny (np. po godzinie 23 następuje 0, po grudniu następuje styczeń). Aby uwzględnić tę cykliczność, zastosowano kodowanie za pomocą funkcji sinusoidalnych:

\begin{itemize}
    \item Dla \texttt{day\_of\_week}: \(\sin\left(\frac{2\pi \cdot \text{day\_of\_week}}{7}\right)\),
    \item Dla \texttt{month}: \(\sin\left(\frac{2\pi \cdot \text{month}}{12}\right)\),
    \item Dla \texttt{hour}: \(\sin\left(\frac{2\pi \cdot \text{hour}}{24}\right)\).
\end{itemize}

Oryginalne zmienne zostały usunięte, a ich zakodowane wersje dodano do zbioru danych. Kodowanie sinusoidalne pozwala modelom lepiej uchwycić cykliczność danych, w przeciwieństwie do kodowania typu one-hot, które zwiększyłoby wymiarowość danych i nie uwzględniało cykliczności.

Następnie, wszystkie zmienne numeryczne niebinarne zostały poddane standaryzacji StandardScaler z biblioteki \texttt{sklearn}. Standaryzacja polega na przekształceniu zmiennych, aby miały średnią 0 i odchylenie standardowe 1. Wartości zmiennych zostały przekształcone według wzoru:
\[ z = \frac{x - \mu}{\sigma}, \]
gdzie \( z \) to wartość po standaryzacji, \( x \) to wartość przed standaryzacją, \( \mu \) to średnia zmiennej, a \( \sigma \) to odchylenie standardowe. 

Standaryzacja jest kluczowym krokiem w preprocessingu danych, ponieważ większość algorytmów uczenia maszynowego zakłada, że dane mają podobną skalę. W przeciwnym razie algorytmy mogą być wrażliwe na różnice w skali zmiennych, co prowadzi do nieoptymalnych wyników. Proces standaryzacji został przeprowadzony osobno dla każdego z okresów spokojnego i niespokojnego osobno, aby uwzględnić różnice w rozkładach danych między tymi okresami. W obrębie każdego okresu parametry standaryzacji (średnia i odchylenie standardowe) obliczono na zbiorze treningowym i zastosowano zarówno do danych treningowych, jak i testowych, aby uniknąć wycieku informacji.

Wartości odstające (np. ekstremalnie wysokie ceny w okresie niespokojnym, takie jak 3812,45 PLN/MWh) nie zostały zmodyfikowane, ponieważ odzwierciedlają rzeczywiste zjawiska rynkowe (np. kryzys energetyczny w 2022 roku). Ich wpływ na modele będzie monitorowany podczas analizy wyników.

    \chapter{Metodologia}
\label{ch:metodologia}

W rozdziale tym przedstawiono metodologię przeprowadzonych badań. Rozdział składa się z dwóch sekcji. W pierwszej przedstawiono metodykę oceny jakości prognoz, a w drugiej omówiono metodykę prognozowania cen energii elektrycznej.

\section{Ocena jakości prognoz}
\label{sec:ocena_jakosci_prognoz}

Ocena jakości modeli prognozowania cen energii elektrycznej jest kluczowym etapem analizy, ponieważ pozwala na porównanie skuteczności różnych podejść. W niniejszej pracy zastosowano następujące popularne metryki oceny: Mean Absolute Error (MAE), Root Mean Squared Error (RMSE), Mean Absolute Percentage Error (MAPE), Symmetric Mean Absolute Percentage Error (sMAPE) oraz \( R^2 \). Wszystkie z tych metryk są omawiane w literaturze porównania skuteczności modeli \cite{en17225797}. W pracy prof. Werona \cite{WERON20141030} podano, że nie ma standardu obliczenia metryk EPF i wspomina o innych metrykach stosowanych przez innych autorów artykułów, między innymi wymieniono - Ważony Średni Błąd Bezwzględny (WMAE), średni błąd dniowy (MDE) i tygodniowy (MWE). Niemniej jednak, w tej pracy skupiono się na tych najszerzej stosowanych metrykach. Każda z tych metryk ma swoje zalety i ograniczenia, których omówienie jest przedstawione poniżej, wraz z ich matematycznymi definicjami i przykładami zastosowania w EPF.

\subsection{Mean Absolute Error (MAE)}
\label{subsec:mae}

Mean Absolute Error jest jedną z najprostszych i najczęściej stosowanych metryk w prognozowaniu szeregów czasowych, w tym w EPF. MAE mierzy średnią wartość bezwzględnych błędów prognoz, co pozwala na ocenę dokładności modelu bez uwzględniania kierunku błędu (nad- lub niedoszacowania).

Matematyczna definicja MAE jest następująca:

\[
\text{MAE} = \frac{1}{n} \sum_{t=1}^{n} \left| y_t - \hat{y}_t \right|
\]

gdzie:
\begin{itemize}
    \item \( y_t \) to rzeczywista cena energii w godzinie \( t \),
    \item \( \hat{y}_t \) to przewidywana cena energii w godzinie \( t \),
    \item \( n \) to liczba obserwacji w zbiorze testowym.
\end{itemize}

MAE jest wyrażane w tej samej jednostce co prognozowane wartości (w omawianym przypadku jest to PLN/MWh), co czyni je łatwym do interpretacji. Na przykład, jeśli MAE wynosi 10 PLN/MWh, oznacza to, że średni błąd prognozy wynosi 10 PLN na każdą megawatogodzinę.

\textbf{Zalety MAE:}
\begin{itemize}
    \item Prosta interpretacja i obliczenia.
    \item Równomierne traktowanie wszystkich błędów, niezależnie od ich kierunku.
\end{itemize}

\textbf{Ograniczenia MAE:}
\begin{itemize}
    \item Nie uwzględnia kwadratu błędów, przez co nie penalizuje większych odchyleń w sposób szczególny, co może być problematyczne w EPF, gdzie duże skoki cen (np. w godzinach szczytu) są istotne.
\end{itemize}

\subsection{Root Mean Squared Error (RMSE)}
\label{subsec:rmse}

Root Mean Squared Error (RMSE) jest kolejną popularną metryką w EPF, która uwzględnia kwadrat błędów, co powoduje większe uwzględnienie większych odchyleń między wartościami rzeczywistymi a przewidywanymi. RMSE jest szczególnie użyteczne w sytuacjach, gdzie duże błędy prognoz mogą mieć poważne konsekwencje ekonomiczne.

Definicja RMSE jest następująca:

\[
\text{RMSE} = \sqrt{\frac{1}{n} \sum_{t=1}^{n} \left( y_t - \hat{y}_t \right)^2}
\]

gdzie:
\begin{itemize}
    \item \( y_t \), \( \hat{y}_t \) i \( n \) mają takie same znaczenie jak w MAE.
\end{itemize}

RMSE jest również wyrażane w jednostkach oryginalnych danych, co ułatwia interpretację. Na przykład, RMSE równe 15 PLN/MWh oznacza, że typowy błąd prognozy (w sensie średniego kwadratu) wynosi 15 PLN na megawatogodzinę.

\textbf{Zalety RMSE:}
\begin{itemize}
    \item Większa wrażliwość na duże błędy, co jest istotne w EPF, gdzie skoki cen mogą być kosztowne.
\end{itemize}

\textbf{Ograniczenia RMSE:}
\begin{itemize}
    \item Wrażliwość na wartości odstające - pojedyncze duże błędy mogą znacząco zawyżyć wartość RMSE.
    \item Mniej intuicyjne w interpretacji niż MAE, ponieważ kwadrat błędów zmienia skalę.
\end{itemize}

\subsection{Mean Absolute Percentage Error (MAPE)}
\label{subsec:mape}

Mean Absolute Percentage Error (MAPE) jest metryką wyrażającą błąd prognozy jako procent rzeczywistej wartości, co czyni ją szczególnie użyteczną w porównaniach między różnymi zbiorami danych lub rynkami o różnych poziomach cen.

Definicja MAPE jest następująca:

\[
\text{MAPE} = \frac{1}{n} \sum_{t=1}^{n} \left| \frac{y_t - \hat{y}_t}{y_t} \right| \times 100
\]

gdzie:
\begin{itemize}
    \item \( y_t \), \( \hat{y}_t \) i \( n \) mają takie same znaczenie jak wcześniej.
\end{itemize}

MAPE jest wyrażane w procentach, co ułatwia interpretację. Na przykład, MAPE równe 5\% oznacza, że średni błąd prognozy wynosi 5\% rzeczywistej ceny. W kontekście RDN, jeśli cena energii wynosi 200 PLN/MWh, a MAPE wynosi 5\%, średni błąd wynosi 10 PLN/MWh.

\textbf{Zalety MAPE:}
\begin{itemize}
    \item Intuicyjna interpretacja w procentach, nie trzeba zastanawiać się nad jednostkami bądź kursami walutowymi.
\end{itemize}

\textbf{Ograniczenia MAPE:}
\begin{itemize}
    \item Problemy z wartościami bliskimi zera - jeśli \( y_t \) jest bardzo małe, co jest możliwe w godzinach nocnych, dzielenie przez \( y_t \) prowadzi do bardzo dużych wartości procentowych, a nawet do błędu matematycznego (dzielenie przez zero).
    \item Asymetria - MAPE bardziej penalizuje niedoszacowania niż przeszacowania, co może prowadzić do nieobiektywnej oceny.
\end{itemize}

\subsection{Symmetric Mean Absolute Percentage Error (sMAPE)}
\label{subsec:smape}

Symmetric Mean Absolute Percentage Error (sMAPE) jest zmodyfikowaną wersją MAPE, która rozwiązuje problem asymetrii i dzielenia przez zero. sMAPE uwzględnia zarówno rzeczywiste, jak i przewidywane wartości w mianowniku, co czyni ją bardziej stabilną w sytuacjach, gdy ceny energii są niskie.

Definicja sMAPE jest następująca:

\[
\text{sMAPE} = \frac{1}{n} \sum_{t=1}^{n} \frac{\left| y_t - \hat{y}_t \right|}{\left( \left| y_t \right| + \left| \hat{y}_t \right| \right) / 2} \times 100
\]

gdzie:
\begin{itemize}
    \item \( y_t \), \( \hat{y}_t \) i \( n \) mają takie same znaczenie jak wcześniej.
\end{itemize}

Podobnie jak MAPE, sMAPE jest wyrażane w procentach. Na przykład, sMAPE równe 4\% oznacza, że średni błąd symetryczny wynosi 4\% średniej wartości rzeczywistej i przewidywanej ceny.

\textbf{Zalety sMAPE:}
\begin{itemize}
    \item Rozwiązuje problem dzielenia przez zero, co jest istotne w EPF, gdzie ceny mogą być bliskie zera.
    \item Symetria - traktuje nad- i niedoszacowania w bardziej zrównoważony sposób niż MAPE.
\end{itemize}

\textbf{Ograniczenia sMAPE:}
\begin{itemize}
    \item Nadal może być wrażliwe na skrajne wartości, choć w mniejszym stopniu niż MAPE.
    \item Interpretacja jest mniej intuicyjna niż w przypadku MAE czy RMSE, ponieważ uwzględnia zarówno \( y_t \), jak i \( \hat{y}_t \) w mianowniku.
\end{itemize}

\subsection{Współczynnik determinacji}
\label{subsec:r2}

Współczynnik determinacji, oznaczany jako \( R^2 \), jest metryką powszechnie stosowaną w analizie regresji i prognozowaniu. \( R^2 \) mierzy, jak dobrze model wyjaśnia zmienność danych rzeczywistych, czyli jaki procent wariancji zmiennej zależnej jest wyjaśniony przez model prognostyczny. Jest to metryka szczególnie użyteczna w ocenie modeli liniowych, ale znajduje zastosowanie również w bardziej złożonych modelach, w celu ogólnej oceny ich dopasowania do danych.

Definicja \( R^2 \) jest następująca:

\[
R^2 = 1 - \frac{\sum_{t=1}^{n} \left( y_t - \hat{y}_t \right)^2}{\sum_{t=1}^{n} \left( y_t - \bar{y} \right)^2}
\]

gdzie:
\begin{itemize}
    \item \( y_t \), \( \hat{y}_t \) i \( n \) mają takie same znaczenie jak wcześniej,
    \item \( n \) to liczba obserwacji w zbiorze testowym.
\end{itemize}

Licznik w wyrażeniu \( \sum_{t=1}^{n} \left( y_t - \hat{y}_t \right)^2 \) to suma kwadratów reszt, czyli całkowity błąd modelu, natomiast mianownik \( \sum_{t=1}^{n} \left( y_t - \bar{y} \right)^2 \) to całkowita suma kwadratów, czyli całkowita wariancja danych względem ich średniej. \( R^2 \) przyjmuje wartości w przedziale od 0 do 1, gdzie:
\begin{itemize}
    \item \( R^2 = 1 \) oznacza, że model idealnie przewiduje wszystkie wartości (błąd wynosi 0),
    \item \( R^2 = 0 \) oznacza, że model nie wyjaśnia żadnej zmienności danych i jest równoważny prostemu modelowi średniej (\( \hat{y}_t = \bar{y} \)).
\end{itemize}

W kontekście EPF, na przykład na RDN, \( R^2 \) równe 0,85 oznaczałoby, że model wyjaśnia 85\% zmienności cen energii.

\textbf{Zalety \( R^2 \):}
\begin{itemize}
    \item Intuicyjna interpretacja - \( R^2 \) jasno wskazuje, jaki procent zmienności danych jest wyjaśniony przez model.
    \item Bez jednostek - umożliwia porównanie modeli na różnych zbiorach danych, niezależnie od skali cen (np. PLN/MWh na RDN vs. EUR/MWh na EEX).
\end{itemize}

\textbf{Ograniczenia \( R^2 \):}
\begin{itemize}
    \item Wrażliwość na przeuczenie - \( R^2 \) może być zawyżone w modelach o dużej liczbie parametrów, szczególnie w przypadku małych zbiorów danych, co może prowadzić do mylnego wniosku o dobrym dopasowaniu modelu.
    \item Brak informacji o kierunku błędów - \( R^2 \) nie rozróżnia, czy model nad- czy niedoszacowuje wartości, co w EPF może być istotne z ekonomicznego punktu widzenia.
\end{itemize}

W niniejszej pracy \( R^2 \) zostanie wykorzystane jako dodatkowa metryka oceny, aby uzupełnić analizę opartą na MAE, RMSE, MAPE, sMAPE.

\section{Wybrane metody weryfikacji zbioru danych}
\label{sec:metody_weryfikacji_zbioru_danych}

Stworzony zbiór danych z cechami objaśniającymi ceny energii elektrycznej należy zweryfikować pod kątem jego skuteczności. W związku z tym zostały wybrane cztery metody prognozowania.

\subsection{Regresja liniowa}

\textbf{Opis metody} \\
Regresja liniowa jest jednym z najprostszych i najczęściej stosowanych modeli statystycznych w analizie zbiorów danych. Zakłada liniową zależność między zmienną zależną, a zestawem zmiennych niezależnych (predyktorów). W kontekście EPF regresja liniowa jest często stosowana jako model bazowy, który pozwala na szybkie uzyskanie prognoz i ocenę wpływu poszczególnych zmiennych na ceny energii. Jej zaletą jest prostota interpretacji oraz niski koszt obliczeniowy, co czyni ją odpowiednią do analizy dużych zbiorów danych, co czyni ją odpowiednią do tej pracy.

\textbf{Wzór modelu} \\
Model regresji liniowej można zapisać jako:
\begin{equation}
y = \beta_0 + \beta_1 x_1 + \beta_2 x_2 + \dots + \beta_p x_p + \epsilon
\end{equation}
gdzie:
\begin{itemize}
    \item \( y \) - zmienna zależna, cena energii elektrycznej
    \item \( \beta_0 \) - wyraz wolny,
    \item \( \beta_1, \beta_2, \dots, \beta_p \) - współczynniki regresji dla zmiennych niezależnych,
    \item \( x_1, x_2, \dots, x_p \) - zmienne niezależne (predyktory, np. zmienne związane z zapotrzebowaniem, cenami paliw czy danymi kalendarzowymi),
    \item \( \epsilon \) - składnik losowy (błąd), zakładany jako \( \epsilon \sim N(0, \sigma^2) \).
\end{itemize}

W macierzowej formie model przyjmuje postać:
\begin{equation}
\mathbf{y} = \mathbf{X} \boldsymbol{\beta} + \boldsymbol{\epsilon}
\end{equation}
gdzie:
\begin{itemize}
    \item \( \mathbf{y} \) - wektor obserwacji zmiennej zależnej,
    \item \( \mathbf{X} \) - macierz projektowa zawierająca wartości zmiennych niezależnych,
    \item \( \boldsymbol{\beta} \) - wektor współczynników regresji,
    \item \( \boldsymbol{\epsilon} \) - wektor błędów.
\end{itemize}

\textbf{Estymacja parametrów} \\
Parametry modelu \( \boldsymbol{\beta} \) są estymowane za pomocą metody najmniejszych kwadratów (OLS), która minimalizuje sumę kwadratów błędów:
\begin{equation}
\min \sum_{i=1}^n (y_i - \hat{y}_i)^2
\end{equation}
gdzie \( \hat{y}_i = \beta_0 + \beta_1 x_{i1} + \dots + \beta_p x_{ip} \) to przewidywana wartość dla \( i \)-tej obserwacji. Rozwiązanie analityczne to:
\begin{equation}
\boldsymbol{\beta} = (\mathbf{X}^T \mathbf{X})^{-1} \mathbf{X}^T \mathbf{y}
\end{equation}

\textbf{Istotne parametry modelu} \\
W niniejszej pracy regresja liniowa została zaimplementowana za pomocą biblioteki \texttt{scikit-learn} w Pythonie. Kluczowe parametry modelu obejmują:
\begin{itemize}
    \item \texttt{fit\_intercept=True}: Włączenie wyrazu wolnego (\( \beta_0 \)).
    \item \texttt{normalize=False}: Brak normalizacji zmiennych przed estymacją.
    \item \texttt{solver='auto'}: Automatyczny wybór algorytmu estymacji, domyślnie OLS.
\end{itemize}

\textbf{Zalety i ograniczenia} \\
Regresja liniowa jest łatwa do interpretacji, ponieważ współczynniki \( \beta_j \) wskazują, o ile zmieni się cena energii przy wzroście zmiennej \( x_j \) o jednostkę (przy założeniu stałości pozostałych zmiennych). Jednak model zakłada liniowe zależności między zmiennymi, co może być ograniczeniem w przypadku bardziej złożonych, nieliniowych wzorców w danych cen energii, szczególnie w okresie niespokojnym.

\subsection{Regresja grzbietowa (Ridge)}

\textbf{Opis metody} \\
Regresja grzbietowa (ang. Ridge Regression) jest rozszerzeniem regresji liniowej, które wprowadza regularyzację L2, aby zapobiec przeuczeniu i poprawić stabilność modelu w przypadku współliniowości między zmiennymi objaśniającymi. W prognozowaniu cen energii regresja grzbietowa jest szczególnie użyteczna, gdy zestaw danych zawiera wiele zmiennych, które mogą być skorelowane. Regularyzacja pozwala na zmniejszenie wpływu mniej istotnych zmiennych, co poprawia generalizację modelu.

\textbf{Wzór modelu} \\
Model regresji grzbietowej opiera się na tej samej zależności liniowej co regresja liniowa:
\begin{equation}
y = \beta_0 + \beta_1 x_1 + \beta_2 x_2 + \dots + \beta_p x_p + \epsilon
\end{equation}

Jednak estymacja parametrów uwzględnia dodatkową karę regularyzacyjną L2. Funkcja kosztu w regresji grzbietowej to:
\begin{equation}
\min \sum_{i=1}^n (y_i - \hat{y}_i)^2 + \lambda \sum_{j=1}^p \beta_j^2
\end{equation}
gdzie:
\begin{itemize}
    \item Pierwsza część (\( \sum_{i=1}^n (y_i - \hat{y}_i)^2 \)) to suma kwadratów błędów, jak w OLS.
    \item Druga część (\( \lambda \sum_{j=1}^p \beta_j^2 \)) to kara L2 na wielkość współczynników \( \beta_j \).
    \item \( \lambda \geq 0 \) - parametr regularyzacji, który kontroluje siłę kary (większe \( \lambda \) oznacza silniejszą regularyzację).
\end{itemize}

W macierzowej formie funkcja kosztu to:
\begin{equation}
\min \|\mathbf{y} - \mathbf{X} \boldsymbol{\beta}\|_2^2 + \lambda \|\boldsymbol{\beta}\|_2^2
\end{equation}

Rozwiązanie analityczne dla parametrów to:
\begin{equation}
\boldsymbol{\beta} = (\mathbf{X}^T \mathbf{X} + \lambda \mathbf{I})^{-1} \mathbf{X}^T \mathbf{y}
\end{equation}
gdzie \( \mathbf{I} \) to macierz jednostkowa.

\textbf{Istotne parametry modelu} \\
Regresja grzbietowa została zaimplementowana w Pythonie za pomocą biblioteki \texttt{scikit-learn}. Kluczowe parametry modelu to:
\begin{itemize}
    \item \texttt{alpha=1.0}: Domyślna wartość parametru regularyzacji \( \lambda \). W pracy metodą empiryczną spróbuje się różnych wartości \texttt{alpha} (np. 0.1, 1.0, 10.0, 100.0) za pomocą walidacji krzyżowej, aby wybrać optymalną.
    \item \texttt{fit\_intercept=True}: Włączenie wyrazu wolnego (\( \beta_0 \)).
    \item \texttt{normalize=False}: Brak normalizacji zmiennych przed estymacją (zmienne przeskalowano wcześniej za pomocą \texttt{StandardScaler}).
    \item \texttt{solver='auto'}: Automatyczny wybór algorytmu (domyślnie Cholesky dla małych zbiorów danych lub SAG dla dużych).
\end{itemize}

\textbf{Zalety i ograniczenia} \\
Regresja grzbietowa jest bardziej odporna na współliniowość i przeuczenie od regresji liniowa, co czyni ją odpowiednią do zestawów danych z dużą liczbą zmiennych objaśniających. Jest to szczególnie przydatne, gdyż w pracy uwzględnione zostają parametry temperatury z całej Polski, które zdecydowanie mają korelację. Jednak, podobnie jak regresja liniowa, zakłada liniowe zależności, co może ograniczać jej skuteczność w modelowaniu bardziej złożonych wzorców, szczególnie w niestabilnych okresach rynkowych.

\subsection{Prophet}

\textbf{Opis metody} \\
Prophet \cite{prophet_doc} to model prognozowania szeregów czasowych opracowany przez Facebooka, zaprojektowany do analizy danych z wyraźną sezonowością i trendami, które mogą ulegać zmianom w czasie. W kontekście prognozowania cen energii elektrycznej (EPF) Prophet jest szczególnie ciekawy ze względu na zdolność do modelowania cyklicznych wzorców jakie występują na tym rynku oraz uwzględniania efektów specjalnych, takich jak święta. Model jest oparty na addytywnym podejściu, które rozkłada szereg czasowy na składowe trendu, sezonowości i efektów dodatkowych. Jego intuicyjna parametryzacja i możliwość automatycznego dopasowania do danych czynią go atrakcyjnym narzędziem w analizie dużych zbiorów danych, takich jak te wykorzystane w niniejszej pracy.

\textbf{Wzór modelu} \\
Prophet modeluje zmienną zależną jako sumę trzech głównych składowych plus składnik losowy:
\begin{equation}
y(t) = g(t) + s(t) + h(t) + r(t) + \epsilon_t
\end{equation}
gdzie:
\begin{itemize}
    \item \( y(t) \) - wartość prognozowana,
    \item \( g(t) \) - składowa trendu, modelująca długoterminowe zmiany w danych,
    \item \( s(t) \) - składowa sezonowości, modelująca cykliczne wzorce (np. dobowe, tygodniowe),
    \item \( h(t) \) - składowa efektów specjalnych takich jak święta,
    \item \( r(t) \) - składowa zmiennych objaśniających, uwzględniająca wpływ dodatkowych regresorów, takich jak zapotrzebowanie czy dane pogodowe,
    \item \( \epsilon_t \) - składnik losowy (błąd), zakładany jako \( \epsilon_t \sim N(0, \sigma^2) \).
\end{itemize}

\textbf{Składowa trendu (\( g(t) \))} \\
Trend w modelu Prophet jest modelowany za pomocą nieliniowej funkcji z punktami zmiany (ang. changepoints), które pozwalają na elastyczne dopasowanie do nagłych zmian w danych. Standardowo używa się funkcji liniowej z punktami zmiany:
\begin{equation}
g(t) = (k + \mathbf{a}(t)^T \boldsymbol{\delta}) t + (m + \mathbf{a}(t)^T \boldsymbol{\gamma})
\end{equation}
gdzie:
\begin{itemize}
    \item \( k \) - współczynnik nachylenia trendu,
    \item \( m \) - wyraz wolny,
    \item \( \mathbf{a}(t) \) - wektor binarny wskazujący punkty zmiany,
    \item \( \boldsymbol{\delta} \) - wektor zmian nachylenia w punktach zmiany,
    \item \( \boldsymbol{\gamma} \) - wektor przesunięć dla ciągłości trendu w punktach zmiany.
\end{itemize}

\textbf{Składowa sezonowości (\( s(t) \))} \\
Sezonowość jest modelowana za pomocą szeregu Fouriera, który aproksymuje cykliczne wzorce:
\begin{equation}
s(t) = \sum_{n=1}^N \left( a_n \cos\left(\frac{2\pi n t}{P}\right) + b_n \sin\left(\frac{2\pi n t}{P}\right) \right)
\end{equation}
gdzie:
\begin{itemize}
    \item \( P \) - okres sezonowości (np. 24 godziny dla sezonowości dobowej, 168 godzin dla tygodniowej),
    \item \( a_n, b_n \) - współczynniki szeregu Fouriera,
    \item \( N \) - liczba składników szeregu (kontrolowana przez parametr \texttt{fourier\_order}).
\end{itemize}

\textbf{Składowa efektów specjalnych (\( h(t) \))} \\
Efekty specjalne, takie jak święta, są modelowane jako:
\begin{equation}
h(t) = \mathbf{Z}(t) \boldsymbol{\kappa}
\end{equation}
gdzie:
\begin{itemize}
    \item \( \mathbf{Z}(t) \) - macierz binarna wskazująca wystąpienie efektów specjalnych (np. 1 dla dni świątecznych, 0 w pozostałych),
    \item \( \boldsymbol{\kappa} \) - wektor efektów dla każdego zdarzenia.
\end{itemize}

\textbf{Składowa zmiennych objaśniających (\( r(t) \))} \\
Zmienne objaśniające, takie jak zapotrzebowanie, dane pogodowe czy bilanse handlowe, są uwzględniane jako dodatkowa już znana składowa liniowa:
\begin{equation}
r(t) = \beta_1 x_1(t) + \beta_2 x_2(t) + \dots + \beta_p x_p(t)
\end{equation}

\textbf{Estymacja parametrów} \\
Parametry modelu (\( k, m, \boldsymbol{\delta}, \boldsymbol{\gamma}, a_n, b_n, \boldsymbol{\kappa}, \beta_1, \dots, \beta_p \)) są estymowane za pomocą maksymalizacji funkcji wiarogodności lub metod bayesowskich. Prophet wykorzystuje algorytm L-BFGS do optymalizacji w trybie domyślnym, co zapewnia szybkie dopasowanie modelu. Punkty zmiany są automatycznie wykrywane, a ich liczba i rozmieszczenie są kontrolowane przez parametry modelu, takie jak \texttt{n\_changepoints} i \texttt{changepoint\_prior\_scale}.

\textbf{Istotne parametry modelu} \\
W niniejszej pracy model Prophet został zaimplementowany w Pythonie za pomocą biblioteki \texttt{prophet}. Dane wejściowe zostały przygotowane w formacie wymaganym przez Prophet, gdzie kolumna \texttt{ds} zawiera znaczniki czasowe (\texttt{timestamp}), a kolumna \texttt{y} zawiera ceny energii (\texttt{fixing\_i\_price}). Kluczowe parametry modelu to:
\begin{itemize}
    \item \texttt{n\_changepoints}: Liczba punktów zmiany trendu, umożliwiająca dopasowanie do potencjalnych zmian w danych (domyślna wartość 25).
    \item \texttt{changepoint\_prior\_scale}: Siła regularyzacji punktów zmiany (domyślna wartość 0.05).
    \item \texttt{yearly\_seasonality=False}: Wyłączenie sezonowości rocznej, ponieważ dane cen energii wykazują głównie sezonowość dobową i tygodniową.
    \item \texttt{weekly\_seasonality=True}: Włączenie sezonowości tygodniowej.
    \item \texttt{daily\_seasonality=True}: Włączenie sezonowości dobowej.
    \item \texttt{fourier\_order}: Liczba składników szeregu Fouriera dla każdej sezonowości (domyślna wartość 10).
    \item \texttt{holidays}: Włączono efekty dni świątecznych na podstawie zmiennej \texttt{is\_holiday} z danych.
    \item Dodatkowe regresory: W przypadku pełnego zestawu danych wszystkie zmienne objaśniające, takie jak zapotrzebowanie, dane pogodowe, bilanse handlowe czy ceny paliw, zostały dodane za pomocą funkcji \texttt{add\_regressor}.
\end{itemize}

\textbf{Zalety i ograniczenia} \\
Prophet jest intuicyjny i dobrze radzi sobie z danymi o wyraźnej sezonowości, co czyni go odpowiednim do modelowania cen energii w stabilnych okresach. Automatyczne wykrywanie punktów zmiany, obsługa efektów specjalnych oraz możliwość włączenia wszystkich zmiennych objaśniających ułatwiają jego stosowanie w praktyce. Jednak model może mieć trudności z modelowaniem bardzo dużych wahań cen, takich jak te obserwowane w okresie niespokojnym (2020-2023), szczególnie jeśli zmienne objaśniające nie w pełni tłumaczą zmienność. Ponadto Prophet zakłada addytywną strukturę szeregu czasowego, co może ograniczać jego zdolność do wychwytywania bardziej złożonych, nieliniowych zależności.

\subsection{Wielowarstwowy perceptron (MLP)}

\textbf{Opis metody} \\
Wielowarstwowy perceptron (MLP) to rodzaj sztucznej sieci neuronowej wykorzystywany w zadaniach uczenia maszynowego. Składa się z warstw neuronów: wejściowej, ukrytych i wyjściowej, które są w pełni połączone.

\textbf{Wzór modelu} \\
MLP przekształca wektor zmiennych wejściowych \( \mathbf{x} = [x_1, x_2, \dots, x_p]^T \) w wartość prognozowaną \( \hat{y} \) (cenę energii, oznaczaną w pracy jako \texttt{fixing\_i\_price}) poprzez sekwencję warstw neuronów. Dla sieci z jedną warstwą ukrytą model można zapisać jako:
\begin{equation}
\hat{y} = f_o\left( \mathbf{w}_o^T \mathbf{h} + b_o \right)
\end{equation}
gdzie:
\begin{itemize}
    \item \( \mathbf{h} = f_h\left( \mathbf{W}_h \mathbf{x} + \mathbf{b}_h \right) \) - wektor aktywacji warstwy ukrytej,
    \item \( \mathbf{x} = [x_1, x_2, \dots, x_p]^T \) - wektor zmiennych objaśniających,
    \item \( \mathbf{W}_h \) - macierz wag między warstwą wejściową a ukrytą,
    \item \( \mathbf{b}_h \) - wektor biasów warstwy ukrytej,
    \item \( f_h(\cdot) \) - funkcja aktywacji warstwy ukrytej (np. \texttt{tanh}),
    \item \( \mathbf{w}_o \) - wektor wag między warstwą ukrytą a wyjściową,
    \item \( b_o \) - bias warstwy wyjściowej,
    \item \( f_o(\cdot) \) - funkcja aktywacji warstwy wyjściowej (dla regresji zazwyczaj liniowa, tj. \( f_o(z) = z \)).
\end{itemize}

Dla sieci z wieloma warstwami ukrytymi proces jest analogiczny, z kolejnymi przekształceniami dla każdej warstwy:
\begin{equation}
\mathbf{h}_k = f_k\left( \mathbf{W}_k \mathbf{h}_{k-1} + \mathbf{b}_k \right), \quad k = 1, 2, \dots, K
\end{equation}
gdzie \( \mathbf{h}_0 = \mathbf{x} \), \( K \) to liczba warstw ukrytych, a \( \mathbf{h}_K \) to wejście do warstwy wyjściowej.

W części analitycznej pracy są przedstawione wyniki z różną ilością warstw ukrytych. 

\textbf{Estymacja parametrów} \\
Parametry modelu (\( \mathbf{W}_k, \mathbf{b}_k \) dla każdej warstwy oraz \( \mathbf{w}_o, b_o \)) są estymowane przez minimalizację funkcji kosztu, czyli średniego błędu kwadratowego (MSE):
\begin{equation}
\text{MSE} = \frac{1}{n} \sum_{i=1}^n \left( y_i - \hat{y}_i \right)^2
\end{equation}
gdzie \( y_i \) to rzeczywista cena energii, a \( \hat{y}_i \) to przewidywana wartość dla \( i \)-tej obserwacji.

Estymacja parametrów odbywa się za pomocą algorytmu wstecznej propagacji błędu (backpropagation) w połączeniu z optymalizatorem, takim jak Adam. Proces treningu polega na iteracyjnym dostosowywaniu wag i biasów w celu zmniejszenia błędu na zbiorze treningowym, z uwzględnieniem walidacji na oddzielnym zbiorze danych w celu uniknięcia przeuczenia.

\textbf{Istotne parametry modelu} \\
W niniejszej pracy model MLP został zaimplementowany w Pythonie za pomocą modułu \texttt{Keras} z biblioteki TensorFlow. Kluczowe parametry modelu, wraz z ich domyślnymi wartościami w bibliotece \texttt{Keras}, obejmują:
\begin{itemize}
    \item \texttt{units}: Liczba neuronów w każdej warstwie ukrytej.
    \item \texttt{activation}: Funkcja aktywacji dla warstw ukrytych (domyślnie \texttt{relu} dla warstw gęstych).
    \item \texttt{optimizer}: Algorytm optymalizacji (domyślnie \texttt{rmsprop} dla modelu sekwencyjnego).
    \item \texttt{learning\_rate}: Szybkość uczenia dla optymalizatora (domyślnie 0.001 dla optimizera \texttt{Adam}).
    \item \texttt{batch\_size}: Rozmiar partii danych w każdej iteracji treningu (domyślnie 32 w metodzie \texttt{fit}).
    \item \texttt{epochs}: Maksymalna liczba epok treningu.
\end{itemize}

\textbf{Zalety i ograniczenia} \\
MLP jest elastycznym modelem zdolnym do wychwytywania nieliniowych zależności w danych, co czyni go odpowiednim do modelowania cen energii w okresach o dużej zmienności. Możliwość dostosowania architektury sieci i hiperparametrów pozwala na optymalizację modelu pod kątem specyfiki danych. Jednak MLP wymaga starannego doboru hiperparametrów i preprocessingu danych, a jego trening jest bardziej kosztowny obliczeniowo niż w przypadku modeli statystycznych. Ponadto model może być podatny na przeuczenie, jeśli liczba warstw lub neuronów jest zbyt duża w stosunku do dostępnych danych.

    \chapter{Estymacja przykładowych modeli}
\label{ch:analiza}

W niniejszym rozdziale przedstawiono wyniki modelowania cen energii za pomocą czterech modeli: regresji liniowej, regresji Ridge, Propheta oraz MLP. Analiza została podzielona na dwa podrozdziały, odpowiadające okresom stabilnemu i niestabilnemu. Zbiór danych został poddany testom w celu oceny skuteczności. W celu uzyskania najlepszych wyników dla modeli przeprowadzono strojenie hiperparametrów.

\section{Okres stabilny}
\label{sec:okres_stabilny}

\subsubsection{Regresja liniowa i Ridge}

Przeanalizowano modele regresji liniowej oraz regresji Ridge na danych z okresu stabilnego. Modele trenowano na danych z lat 2016--2018, a testowano na danych z 2019 roku.

Wyniki dla pełnego zbioru o 60 parametrach wejściowych przedstawiono w tabeli~\ref{tab:linear_ridge_results_full}. Regresja Ridge osiągnęła lepsze wyniki niż regresja liniowa, co jest widoczne w tabeli poniżej na podstawie przedstawionych wcześniej metryk ~\ref{sec:ocena_jakosci_prognoz}. Wynika to z powodu regularyzacji L2 zastosowanej w modelu Ridge. 

\begin{table}[H]
    \centering
    \caption{Wyniki regresji liniowej i grzebietowej dla pełnego zbioru danych w okresie stabilnym (2019).}
    \label{tab:linear_ridge_results_full}
    \begin{tabular}{|l|ccccc|}
        \hline
        \textbf{Model} & \textbf{MAE} & \textbf{RMSE} & \textbf{MAPE (\%)} & \textbf{sMAPE (\%)} & \textbf{\(R^2\)} \\
        \hline
        Regresja liniowa & 15.18 & 19.79 & 7.25 & 7.16 & 0.8413 \\
        Regresja Ridge   & 15.09 & 19.64 & 7.19 & 7.09 & 0.8437 \\
        \hline
    \end{tabular}
\end{table}

Najlepsza wartość hiperparametru \(\alpha\) w regresji Ridge wyniosła 500.0. Wysoka wartość \(\alpha = 500.0\) sugeruje, że w pełnym zbiorze danych występuje istotna współliniowość między zmiennymi objaśniającymi. Potwierdza to również analiza macierzy korelacji ~\ref{fig:correlation_plot}.

Następnie przeprowadzono analizę skróconego zbioru danych opisanego w~\ref{sec:shortened_dataset}. Wyniki dla skróconego zbioru danych są przedstawione w tabeli~\ref{tab:linear_ridge_results_short}, wraz z różnicami w metrykach względem pełnego zbioru danych. Różnice w metrykach wskazują, że dodatkowe zmienne w pełnym zbiorze danych wnoszą informację, mimo niższego poziomu korelacji ze zmienną objaśnianą. Różnice metryk nie są bardzo duże (np. MAPE różni się o 0.16\% dla regresji Ridge), co może sugerować, że skrócony zbiór danych nadal zawiera najważniejsze zmienne objaśniające.

\begin{table}[h]
    \centering
    \caption{Wyniki regresji liniowej i Ridge dla skróconego zbioru danych w okresie stabilnym (2019) wraz z różnicami względem pełnego zbioru.}
    \label{tab:linear_ridge_results_short}
    \begin{tabular}{|l|ccccc|c|}
        \hline
        \textbf{Model} & \textbf{MAE} & \textbf{RMSE} & \textbf{MAPE (\%)} & \textbf{sMAPE (\%)} & \textbf{\(R^2\)} & \textbf{Różnica MAPE (\%)} \\
        \hline
        Regresja liniowa & 15.61 & 20.31 & 7.33 & 7.32 & 0.8328 & +0.08 \\
        Regresja Ridge   & 15.67 & 20.40 & 7.35 & 7.34 & 0.8314 & +0.16 \\
        \hline
    \end{tabular}
\end{table}

W celu potencjalnego polepszenia wyników zastosowano logarytmizację zmiennej wyjściowej (\texttt{fixing\_i\_price}), co miało na celu zmniejszenie skośności rozkładu cen i poprawę dopasowania modelu. Wyniki z logarytmizacją przedstawiono w tabeli~\ref{tab:linear_ridge_results_log}.

\begin{table}[h]
    \centering
    \caption{Wyniki regresji liniowej i Ridge z logarytmizacją dla okresu stabilnego (2019).}
    \label{tab:linear_ridge_results_log}
    \begin{tabular}{|l|ccccc|}
        \hline
        \textbf{Model i zbiór danych} & \textbf{MAE} & \textbf{RMSE} & \textbf{MAPE (\%)} & \textbf{sMAPE (\%)} & \textbf{\(R^2\)} \\
        \hline
        Regresja liniowa (pełny)   & 21.63 & 28.36 & 9.85 & 9.16 & 0.6736 \\
        Regresja Ridge (pełny)     & 19.99 & 26.06 & 9.21 & 8.65 & 0.7244 \\
        Regresja liniowa (skrócony) & 26.36 & 33.07 & 11.95 & 10.94 & 0.5563 \\
        Regresja Ridge (skrócony)   & 23.29 & 29.38 & 10.69 & 9.89 & 0.6498 \\
        \hline
    \end{tabular}
\end{table}

Logarytmizacja nie przyniosła spodziewanych korzyści i pogorszyła wyniki we wszystkich metrykach. Największe pogorszenie zaobserwowano dla skróconego zbioru danych, gdzie MAPE dla regresji liniowej wzrosło do 11.95\%, a \(R^2\) spadło do 0.5563. Przyczyną jest najprawdopodobniej fakt, że logarytmizacja wprowadziła niepotrzebne nieliniowości, które utrudniły dopasowanie modeli liniowych. Dodatkowo, odwrócenie transformacji logarytmicznej może amplifikować błędy predykcji, co wpłynęło na wzrost RMSE i MAE.

Regresja Ridge lepiej przewiduje ceny w porównaniu do regresji liniowej, co potwierdzają wyniki metryk. Poniżej umieszczam wykres rzeczywistych i przewidywanych wartości dla regresji Ridge w okresie stabilnym. Załączony został wykres z pierwszego kwartału 2019 roku, ponieważ wykres z całego 2019 nie jest czytelny z powodu dużej liczby punktów.

\begin{figure}[H]
    \centering
    \includegraphics[width=1.0\textwidth]{../../plots/predicts/ridge_predictions_full_q1_2019.png}
    \caption{Porównanie rzeczywistych i przewidywanych wartości cen energii dla regresji Ridge w okresie stabilnym. Opracowanie własne.}
    \label{fig:ridge_predictions_full_stable_period}
\end{figure}

Widoczne na wykresie jest to, że model nie zawsze nadążą za dużymi skokami cen energii, prawdopodobnie z powodu ich nieliniowości. Z kolei w przypadku okresów mniejszych zmian i stabilniejszych cen, model oczekuje większej zmienności, co prowadzi do przeszacowania prognoz. Poniżej załączony jest wykres błędów dla regresji Ridge.

\begin{figure}[H]
    \centering
    \includegraphics[width=1.0\textwidth]{../../plots/predicts/errors_over_time_Ridge_full_stable_period.png}
    \caption{Błędy prognoz dla regresji Ridge w okresie stabilnym. Opracowanie własne.}
    \label{fig:ridge_errors_full_stable_period}
\end{figure}

Błędy prognoz dla regresji Ridge w okresie stabilnym są rozproszone wokół zera, co sugeruje, że model dobrze radzi sobie z przewidywaniem cen energii elektrycznej. Wartości błędów przeważnie nie przekraczają 50 PLN/MWh. Widoczne są szczyty przekraczające 100 PLN/MWh, które mogą być wynikiem dużych skoków cen energii elektrycznej. Poniżej przedstawiony jest histogram reszt, żeby zobrazować rozkład błędów prognoz.

\begin{figure}[H]
    \centering
    \includegraphics[width=1.0\textwidth]{../../plots/predicts/residuals_histogram_Ridge_full_stable_period.png}
    \caption{Histogram reszt dla regresji Ridge w okresie stabilnym. Opracowanie własne.}
    \label{fig:ridge_residuals_stable_period}
\end{figure}

Histogram posiada szczyt w okolicy zera i większość błędów skupia się w okolicy zera. Rozkład reszt jest zbliżony do normalnego bez istotnych odchyleń. Wartości reszt są rozproszone w okolicy zera, co sugeruje, że model dobrze radzi sobie z przewidywaniem cen energii elektrycznej w okresie stabilnym. Widoczne na wykresie ~\ref{fig:ridge_residuals_stable_period} szczyty dochodzące do 100 PLN/MWh nie są widoczne na histogramie, co sugeruje, że są to pojedyncze przypadki dużych błędów prognozowania. Wartości tych błędów nie mają wyraźnych wzorców czasowych ani sezonowych. Pojawienie się dużych błędów wynika z fluktuacji cen energii elektrycznej, wynikających z czynników nie wytłumaczalnych przez zbiór zmiennych objaśniających lub czynników zewnętrznych.

\subsubsection{Prophet}
Model Prophet został skonfigurowany z trybem addytywnym (\texttt{seasonality\_mode='additive'}), ponieważ wstępne testy wykazały, że tryb multiplikatywny (\texttt{seasonality\_mode='multiplicative'}) działa dużo gorzej w przypadku zebranych danych. Tryb multiplikatywny zakłada proporcjonalne skalowanie efektów sezonowych i trendów względem wartości zmiennej objaśnianej, co nie jest odpowiednie dla cen energii elektrycznej, gdzie efekty sezonowe (np. różnice między dniami roboczymi a weekendami) mają raczej charakter addytywny, a nie proporcjonalny.

Testowano różne kombinacje pozostałych hiperparametrów opisanych w rozdziale \ref{subsec:prophet}, aby znaleźć optymalne ustawienia. Najlepsze rezultaty uzyskano dla kombinacji (\texttt{changepoint\_prior\_scale=0.100}, \texttt{seasonality\_prior\_scale=20.0}, \texttt{holidays\_prior\_scale=0.1}), gdzie gdzie MAE wyniosło 15.60, RMSE 20.17, MAPE 7.42\%, a sMAPE 7.33\%. Kombinacja ta charakteryzuje się umiarkowaną elastycznością trendu (\texttt{changepoint\_prior\_scale=0.100}), co pozwala modelowi wychwytywać zmiany w cenach energii (np. stopniowe wzrosty), oraz wysoką elastycznością sezonowości (\texttt{seasonality\_prior\_scale=20.0}), co dobrze oddaje zmienne wzorce dzienne i tygodniowe. Niski wpływ świąt (\texttt{holidays\_prior\_scale=0.1}) sugeruje, że polskie święta mają ograniczony wpływ na ceny energii w tym okresie. Pełne wyniki dla pełnego zbioru danych przedstawiono w tabeli~\ref{tab:prophet_results_combined_stable}.

\begin{table}[H]
    \centering
    \caption{Wyniki modelu Prophet dla pełnego i skróconego zbioru danych w okresie stabilnym (2019).}
    \label{tab:prophet_results_combined_stable}
    \begin{tabular}{|l|ccccc|}
        \hline
        \textbf{Zbiór danych} & \textbf{MAE} & \textbf{RMSE} & \textbf{MAPE (\%)} & \textbf{sMAPE (\%)} & \textbf{\(R^2\)} \\
        \hline
        Pełny     & 15.60 & 20.17 & 7.42 & 7.33 & 0.835011 \\
        Skrócony  & 17.75 & 22.85 & 8.17 & 8.32 & 0.788273 \\
        \hline
    \end{tabular}
\end{table}

Pełny zbiór danych znowu osiągnął lepsze wyniki od zbioru skróconego. Różnice są większe, niż w przypadku regresji liniowej i Ridge, co sugeruje, że dodatkowe zmienne w pełnym zbiorze danych mają większy wpływ na prognozy modelu Prophet.

Ponizej przedstawiam wykresy rzeczywistych i przewidywanych wartości dla modelu Prophet dla pierwszego kwartału 2019 roku. Wykres jest bardzo podobny do wykresu \ref{fig:ridge_predictions_full_stable_period} dla regresji Ridge.

\begin{figure}[H]
    \centering
    \includegraphics[width=1.0\textwidth]{../../plots/predicts/Prophet_predictions_stable_Q1.png}
    \caption{Porównanie rzeczywistych i przewidywanych wartości cen energii dla modelu Prophet w okresie stabilnym. Opracowane własne}
    \label{fig:prophet_predictions_stable_period}
\end{figure}

Wykres błędów prognoz dla modelu Prophet w okresie stabilnym jest również podobny do wykresu regresji Ridge~\ref{fig:ridge_errors_full_stable_period}.

\begin{figure}[H]
    \centering
    \includegraphics[width=1.0\textwidth]{../../plots/predicts/errors_over_time_Prophet_full_stable_period.png}
    \caption{Błędy prognoz dla modelu Prophet w okresie stabilnym.}
    \label{fig:prophet_errors_full_stable_period}
\end{figure}

Histogram reszt dla modelu Prophet wykazuje podobieństwo do histogramu reszt dla regresji Ridge~\ref{fig:ridge_residuals_stable_period}, ale lepiej wyjaśnia drobne różnice. Ma niższy szczyt w okolicy zera, ale łagodniejsze zejście w kierunku wartości skrajnych.

\begin{figure}[H]
    \centering
    \includegraphics[width=1.0\textwidth]{../../plots/predicts/residuals_histogram_Prophet_full_stable_period_comb_1.png}
    \caption{Histogram reszt dla modelu Prophet w okresie stabilnym. Opracowanie własne.}
    \label{fig:prophet_residuals_stable}
\end{figure}

Największe błędy prognoz występujące w obu modelach są podobne. Największym wspólnym błędem prognoz dla modeli regresji Ridge oraz Prophet jest przeszacowanie ceny energii elektrycznej w dniu 10 Marca o godzinie 5 rano, gdzie rzeczywista cena wynosi 64 PLN/MWh, gdzie prognozy wynoszą 129 PLN/MWh oraz 131 PLN/MWh dla modelu Ridge i Prophet. Prawdopodobnie zmienne objaśniające sygnalizują wzrost cen, która nie miała miejsca, co prowadzi do przeszacowania prognoz.

Analizując wyniki modelu Prophet w odniesieniu do wyników regresji liniowej i grzbietowej, można stwierdzić, że model Prophet osiąga porównywalne wyniki. Wynik MAPE jest o 0.23\% wyższy od regresji Ridge, co sugeruje, że model Prophet nie jest w stanie przewidzieć cen energii elektrycznej lepiej. Natomiast czas wykonania programu w języku programowania \texttt{Python} dla modelu Prophet wynosił ponad 40 sekund, gdzie czas wykonania dla regresji Ridge wynosił 5. W kontekście analizy zebranego zbioru danych, należy zauważyć, że model Prophet nie przynosi żadnych korzyści w porównaniu do regresji Ridge lub liniowej w przypadku prognozowania w okresie stabilnym.

\subsubsection{MLP}

W ramach badania skuteczności modelu MLP w okresie stabilnym, obejmującym rok 2019, przetestowano różne konfiguracje sieci neuronowych. Najlepsze wyniki uzyskano dla architektury składającej się z pięciu warstw ukrytych, o strukturze z odpowiednio 64, 64, 32, 16 , 8 neuronami w kolejnych warstwach. Model został zaprojektowany w sposób sekwencyjny, gdzie każda warstwa ukryta korzystała z funkcji aktywacji ReLU, a dodatkowo zastosowano regularyzację L2 z parametrem 0,1, aby zapobiec przeuczeniu. Między warstwami dodano mechanizm Dropout z wartością 0,2, który losowo dezaktywuje część neuronów podczas treningu, co pomaga w uogólnianiu modelu. Ostatnia warstwa wyjściowa zawierała jeden neuron, odpowiadający za przewidywanie wartości ceny energii.

Model został skompilowany z użyciem optymalizatora Adam, skonfigurowanego z bardzo niską szybkością uczenia na poziomie 0,0001, co pozwoliło na stopniowe i stabilne dostosowywanie wag. Jako funkcję straty wybrano średni błąd kwadratowy (MSE), która dobrze mierzy różnice między przewidywaniami a rzeczywistymi wartościami. Trening modelu przeprowadzono na danych z lat 2016-2018, z maksymalną liczbą pięciuset epok i rozmiarem partii wynoszącym sto dwadzieścia osiem, co umożliwiło efektywne przetwarzanie danych w mniejszych porcjach.

W tabeli poniżej przedstawiam wyniki dla pełnego i skróconego zbiorów danych. 

\begin{table}[H]
    \centering
    \caption{Wyniki modelu MLP dla pełnego i skróconego zbioru danych w okresie stabilnym (2019).}
    \label{tab:mlp_results_combined_stable}
    \begin{tabular}{|l|cccccc|}
        \hline
        \textbf{Zbiór danych} & \textbf{MAE} & \textbf{RMSE} & \textbf{MAPE (\%)} & \textbf{sMAPE (\%)} & \textbf{\(R^2\)} \\
        \hline
        Pełny     & 17.09 & 22.75 & 8.43 & 7.87 & 0.79 \\
        Skrócony  & 34.55 & 53.58 & 14.87 & 13.56 & -0.16 \\
        \hline
    \end{tabular}
\end{table}

Uzyskane wyniki dla zbioru skróconego są znacznie gorsze niż dla zbioru pełnego, co sugeruje, że model MLP nie jest w stanie dobrze przewidzieć cen energii elektrycznej na podstawie ograniczonej liczby zmiennych objaśniających. Wartości MAPE są o ponad procent wyższe niż w przypadku regresji liniowej i Ridge, co sugeruje, że model MLP nie jest w stanie przewidzieć cen energii elektrycznej lepiej niż modele statystyczne. Inne zestawy parametrów lub architektury sieci neuronowej nie przyniosły lepszych wyników.

\begin{figure}[H]
    \centering
    \includegraphics[width=1.0\textwidth]{../../plots/mlp1/mlp_predictions_full_stable_q1_(64, 64, 32, 16, 8).png}
    \caption{Porównanie rzeczywistych i przewidywanych wartości cen energii dla modelu MLP w okresie stabilnym. Opracowanie własne.}
    \label{fig:mlp_predictions_stable_period}
\end{figure}

Wykres~\ref{fig:mlp_predictions_stable_period} nieco różni się od analogicznych wykresów dla modeli statystycznych. Prognozy dobrze podążają za rzeczywistymi wartościami, ale widać, że model MLP ma tendencję do przeszacowywania cen energii elektrycznej w momentach wzrostów ceny. Model oczekuje na większe wzrosty, które nie następują.\newline
Poniżej przedstawione są wykresy błędów prognoz dla modelu MLP oraz histogram reszt dla okresu stabilnego. 

\begin{figure}[H]
    \centering
    \includegraphics[width=1.0\textwidth]{../../plots/mlp1/errors_over_time_(64, 64, 32, 16, 8)_Stabilny_2019_pełny.png}
    \caption{Błędy prognoz dla modelu MLP w okresie stabilnym. Opracowanie własne.}
    \label{fig:mlp_errors_stable_period}
\end{figure}

\begin{figure}[H]
    \centering
    \includegraphics[width=1.0\textwidth]{../../plots/mlp1/mlp_errors_histogram_full_stable_(64, 64, 32, 16, 8).png}
    \caption{Histogram reszt dla modelu MLP w okresie stabilnym. Opracowanie własne.}
    \label{fig:mlp_residuals_stable_period}
\end{figure}

Wykres~\ref{fig:mlp_errors_stable_period} błędów prognoz oscyluje dookoła zera poza okresem w kwietniu, gdzie większość błędów ma wartości ujemne. Widoczne jest, że model MLP ma tendencję do zbyt niskiego oszacowania cen energii elektrycznej w tym okresie. Wartości błędów są znacznie większe niż w przypadku modeli statystycznych, co sugeruje, że model MLP nie radzi sobie dobrze z przewidywaniem cen energii elektrycznej w okresie stabilnym.

Histogram reszt~\ref{fig:mlp_residuals_stable_period} potwierdza te założenia, gdyż mniej przypomina rozkład normalny od poprzednich histogramów. Ma wyższy szczyt, ale jest przesunięty od zera w stronę wartości dodatnich. Większa liczba błędów skupia się w okolicy zera w porównaniu do histogramu reszt dla regresji Ridge lub Prophet.

\section{Okres niestabilny}
\label{sec:okres_niestabilny}

Okres niestabilny obejmuje lata 2020-2023, w których ceny energii elektrycznej były znacznie bardziej zmienne niż w okresie stabilnym. W związku z tym, okres ten może być bardziej wymagający dla modeli prognozujących.

\subsubsection{Regresja liniowa i Ridge}

W analizie wyników regresji liniowej i Ridge dla okresu niestabilnego (2023) dla pełnego i skróconego zbioru danych obserwuje się zbliżone wartości metryk, co wskazuje na ograniczone różnice w skuteczności obu metod w tym okresie. Tabela~\ref{tab:linear_regression_results} przedstawia szczegółowe wyniki dla regresji liniowej i Ridge. Zbiór skrócony podobnie do okresu stabilnego nie przyniósł znaczących różnic w metrykach, ale dodatkowe zmienne w pełnym zbiorze danych lekko poprawiły wyniki niewielkim kosztem obliczeniowym. 

Wysokie wartości MAPE (powyżej 179\%) sugerują problem z wartościami bliskimi zeru ~\ref{subsec:mape}, które pojawiają się w zbiorze danych z okresu niestabilnego. Z tego powodu, MAPE może nie być najlepszą metryką do oceny skuteczności modeli na zbiorze okresu niestabilnego. Z tego powodu większej uwadze poświęcono metrykom MAE, RMSE, sMAPE oraz \(R^2\). Wartości MAE i RMSE są stosunkowo wysokie, co wskazuje na duże błędy prognozowania w jednostkach absolutnych. Błąd na poziomie 59.6 PLN/MWh dla MAE oznacza, że prognozy różnią się średnio o 59.6 PLN od rzeczywistych wartości. Może to prowadzić do znacznych strat finansowych, szczególnie w przypadku dużych transakcji. Wartości sMAPE są dwukrotnie wyższe niż w przypadku okresu stabilnego, co sugeruje, że modele mają trudności z przewidywaniem cen energii w okresach dużej zmienności. Pomimo tego, \(R^2\) nie wzrosło znacząco względem okresu stabilnego, co sugeruje, że modele nadal dobrze wyjaśniają zmienność cen energii, mimo dużych błędów prognozowania. 

Najlepsza wartość hiperparametru \(\alpha\) w regresji Ridge dla okresu niestabilnego wyniosła 0.1, co sugeruje, że w tym przypadku współliniowość między zmiennymi objaśniającymi nie jest tak istotna jak w przypadku pełnego zbioru danych w okresie stabilnym. Wartość ta jest znacznie niższa niż w przypadku pełnego zbioru danych w okresie stabilnym (500.0), co może sugerować, że w okresie niestabilnym modele są bardziej elastyczne.

\begin{table}[H]
    \centering
        \caption{Wyniki metryk dla regresji liniowej i Ridge w okresie niestabilnym (2023). Opracowanie własne.}
        \label{tab:linear_regression_results}
        \begin{tabular}{|l|c|c|c|c|c|}
            \hline
            \textbf{Model} & \textbf{MAE} & \textbf{RMSE} & \textbf{MAPE} & \textbf{sMAPE} & \textbf{\(R^2\)} \\
            \hline
            Regresja liniowa (pełny zbiór) & 59.61 & 77.73 & 179.38 & 16.51 & 0.8111 \\
            Regresja liniowa (skrócony zbiór) & 59.76 & 78.07 & 188.24 & 16.68 & 0.8095 \\
            Regresja Ridge (pełny zbiór) & 59.63 & 77.75 & 179.79 & 16.52 & 0.8110 \\
            Regresja Ridge (skrócony zbiór) & 59.78 & 78.08 & 186.22 & 16.69 & 0.8094 \\
            \hline
        \end{tabular}
\end{table}

Wartości metryk są niemal identyczne dla obu modeli, co sugeruje, że w tym przypadku regularyzacja L2 nie przynosi znaczących korzyści.\newline
Dla dobrego zobrazowania prognoz na wykresie wybrano okres o największej zmienności cen w ramach 2023 roku, czyli od 1 września do 31 października. Wartości prognoz dla regresji Ridge w porównaniu do rzeczywistych cen energii elektrycznej przedstawiono na rysunku~\ref{fig:ridge_predictions_full_sep_oct_2023}. Wykres reszt dla modelu Ridge w okresie niestabilnym przedstawiono na rysunku~\ref{fig:residuals_nonstable_histogram_ridge}.

\begin{figure}[H]
    \centering
    \includegraphics[width=1.0\textwidth]{../../plots/predicts/ridge_predictions_full_sep_oct_2023.png}
    \caption{Prognozy modelu Ridge w porównaniu do rzeczywistych cen energii w okresie niestabilnym (2023). Opracowanie własne.}
    \label{fig:ridge_predictions_full_sep_oct_2023}
\end{figure}

\begin{figure}[H]
    \centering
    \includegraphics[width=1.0\textwidth]{../../plots/predicts/errors_over_time_Ridge_non_stable_period.png}
    \caption{Błędy prognoz dla modelu Ridge w okresie niestabilnym (2023). Opracowanie własne.}
    \label{fig:ridge_errors_full_sep_oct_2023}
\end{figure}

Na wykresie~\ref{fig:ridge_predictions_full_sep_oct_2023} widać, że model jest skłonny przeszacowywać zmienność cen zarówno w górę, jak i w dół. Błędy prognoz dla modelu Ridge w okresie niestabilnym są znacznie większe niż w przypadku okresu stabilnego. Większość błędów oscyluje się od -100 do 100 PLN/MWh. Największy szczyt można zobaczyć na początku okresu testowego, gdzie obserwuje się szczyt powyżej 400 PLN/MWh. Wartości prognoz są znacznie bardziej rozproszone niż w przypadku okresu stabilnego. Histogram reszt dla modelu Ridge w okresie niestabilnym przedstawiono na rysunku~\ref{fig:residuals_nonstable_histogram_ridge}.

\begin{figure}[H]
    \centering
    \includegraphics[width=1.0\textwidth]{../../plots/predicts/residuals_histogram_Ridge_not_stable_period.png}
    \caption{Histogram reszt dla modelu Ridge w okresie niestabilnym (2023). Opracowanie własne.}
    \label{fig:residuals_nonstable_histogram_ridge}
\end{figure}

Na podstawie histogramu reszt można zauważyć, że rozkład reszt jest podobny do rozkładu normalnego i ma szczyt w okolicy zera. Porównując histogram reszt z histogramem reszt~\ref{fig:ridge_residuals_stable_period} dla okresu stabilnego, można zauważyć, że histogram reszt w okresie niestabilnym ma szerszy rozkład z większymi ogonami, co potwierdza większe błędy prognozowania w wartościach absolutnych. Histogram reszt jest lekko przesunięty w kierunku wartości dodatnich, co sugeruje, że model ma tendencję do przeszacowywania cen.

\subsubsection{Prophet}

Model Prophet lekko poprawił wyniki w porównaniu do regresji liniowej i Ridge. Najlepszymi parametrami do okresu niestabilnego okazały się (\texttt{changepoint\_prior\_scale=0.001}, \texttt{seasonality\_prior\_scale=50.0}, \texttt{holidays\_prior\_scale=0.1}). Niska wartość {\texttt{changepoint\_prior\_scale}} sugeruje, że model preferuje bardziej stabilne trendy, unikając nadmiernego dopasowania do szumów w danych treningowych. Zwiększona wartość {\texttt{seasonality\_prior\_scale}} pozwala modelowi lepiej uchwycić sezonowe wzorce w danych, co jest istotne w przypadku cen energii elektrycznej, które mogą wykazywać silne sezonowe fluktuacje. Wartość {\texttt{holidays\_prior\_scale}} jest taka sama jak w przypadku okresu stabilnego, co sugeruje, że wpływ świąt na ceny energii wciąż nie jest istotny. Wyniki dla pełnego zbioru danych przedstawiono w tabeli~\ref{tab:prophet_results_combined_nonstable}.

\begin{table}[H]
    \centering
    \caption{Wyniki modelu Prophet dla pełnego i skróconego zbioru danych w okresie niestabilnym (2023).}
    \label{tab:prophet_results_combined_nonstable}
    \begin{tabular}{|l|ccccc|}
        \hline
            \textbf{Zbiór danych} & \textbf{MAE} & \textbf{RMSE} & \textbf{MAPE (\%)} & \textbf{sMAPE (\%)} & \textbf{\(R^2\)} \\
            Pełny     & 57.87 & 74.39 & 163.57 & 16.35 & 0.827021 \\
            Skrócony  & 61.10 & 78.56 & 202.72 & 18.17 & 0.80709 \\
            \hline
    \end{tabular}
\end{table}

Wartość parametru \texttt{changepoint\_prior\_scale} jest najbardziej znacząca w procesie doboru najlepszych hiperparametrów. Zwiększenie wartości tego parametru do 0.1 prowadzi do znacznego pogorszenia wyników na poziomie \textbf{MAE = 138} oraz \textbf{sMAPE = 20.0\%}. Oznacza to, że model zaczyna być bardziej podatny na wykrywanie zmiany w trendzie i próbuje dopasować się do lokalnych fluktuacji. Zmiana innych hiperparametrów zwiększa sMAPE o 1\%.

Wynik modelu prophet w porównaniu z regresją Ridge jest nieznacząco lepszy z punktu widzenia sMAPE, ale średni błąd MAE jest o 1.76 PLN/MWh niższy, co może być bardzo istotne z punktu finansowego w przypadku dużych transakcji. Wartości RMSE są również niższe, co sugeruje, że model Prophet lepiej radzi sobie z przewidywaniem cen energii elektrycznej w okresie niestabilnym.

\begin{figure}[H]
    \centering
    \includegraphics[width=1.0\textwidth]{../../plots/predicts/Prophet_predictions_unstable_sep_pełny_comb_1.png}
    \caption{Porównanie rzeczywistych i przewidywanych wartości cen energii dla modelu Prophet w okresie niestabilnym. Opracowanie własne.}
    \label{fig:prophet_predictions_non_stable_period}
\end{figure}

Wykres predykcji modelu Prophet w okresie niestabilnym jest podobny do wykresu regresji Ridge ~\ref{fig:ridge_predictions_full_sep_oct_2023}. Widać, że model przeszacowuje ceny energii elektrycznej w okresach dużej zmienności, co prowadzi do dużych błędów prognozowania.\newline
Poniżej przedstawiono wykresy błędów prognoz oraz histogram reszt dla modelu Prophet w okresie niestabilnym.

\begin{figure}[H]
    \centering
    \includegraphics[width=1.0\textwidth]{../../plots/predicts/errors_over_time_Prophet_non_stable.png}
    \caption{Błędy prognoz dla modelu Prophet w okresie niestabilnym. Opracowanie własne.}
    \label{fig:prophet_errors_non_stable_period}
\end{figure}

\begin{figure}[H]
    \centering
    \includegraphics[width=1.0\textwidth]{../../plots/predicts/residuals_histogram_Prophet_unstable_pełny_comb_1.png}
    \caption{Histogram reszt dla modelu Prophet w okresie niestabilnym. Opracowanie własne.}
    \label{fig:prophet_residuals_non_stable}
\end{figure}

Histogram dla modelu Prophet w okresie niestabilnym jest bardziej przesunięty w kierunku wartości dodatnich niż histogram dla regresji Ridge~\ref{fig:residuals_nonstable_histogram_ridge}. Histogram wykazuje większe odchylenia od rozkładu normalnego, i nie posiada szczytu w zerze. Natomiast ogon histogramu po prawej stronie jest krótszy. To jest prawdopodobnie powodem, że model Prophet ma lepsze wyniki od regresji Ridge. 

\subsubsection{MLP}

Dla uzyskania najlepszych wyników dla modelu MLP w okresie niestabilnym, liczbę epok zwiększono do 1000, a rozmiar partii zmniejszono do 64. Wartości hiperparametrów pozostały takie same jak w przypadku okresu stabilnego. Szybkość uczenia została zwiększona do 0.001, co pozwoliło na szybsze dostosowywanie wag. Po raz pierwszy wyniki skróconego zestawu parametrów przewyższyły wyniki pełnego zbioru danych we wszystkich metrykach. Sugeruje to, że model MLP jest bardziej podatny na nadmierne dopasowanie do danych treningowych w przypadku pełnego zbioru danych.\newline
Wartości metryk dla pełnego i skróconego zbioru danych przedstawiono w tabeli~\ref{tab:mlp_results_combined_nonstable}.

\begin{table}[H]
    \centering
    \caption{Wyniki modelu MLP dla pełnego i skróconego zbioru danych w okresie niestabilnym (2023).}
    \label{tab:mlp_results_combined_nonstable}
    \begin{tabular}{|l|cccccc|}
        \hline
        \textbf{Zbiór danych} & \textbf{MAE} & \textbf{RMSE} & \textbf{MAPE (\%)} & \textbf{sMAPE (\%)} & \textbf{\(R^2\)} \\
        Pełny     & 98.85 & 126.89 & 1361.22 & 22.21 & 0.50 \\
        Skrócony  & 66.08 & 88.46 & 1702.13 & 17.38 & 0.76 \\
        \hline
    \end{tabular}
\end{table}

Skrócony zestaw parametrów wyjaśnił 76\% zmienności cen energii elektrycznej, co jest znacznie lepszym wynikiem niż w przypadku pełnego zbioru danych. Jest to jednak wciąż gorszy wynik niż w przypadku wcześniejszych modeli. Wartości absolutne MAE i sMAPE są przykładowo o 8.21 PLN/MWh i 1.03\% gorsze niż w przypadku modelu Prophet.\newline
Wykres rzeczywistych i przewidywanych wartości dla modelu MLP w okresie niestabilnym dla danych skróconych przedstawiono na rysunku~\ref{fig:mlp_predictions_non_stable_period}.

\begin{figure}[H]
    \centering
    \includegraphics[width=1.0\textwidth]{../../plots/mlp2/mlp_predictions_short_unstable_aug_sep_(64, 64, 32, 16, 8).png}
    \caption{Porównanie rzeczywistych i przewidywanych wartości cen energii dla modelu MLP w okresie niestabilnym. Opracowanie własne.}
    \label{fig:mlp_predictions_non_stable_period}
\end{figure}

Wykres potwierdza gorsze wyniki modelu MLP w porównaniu do modelu Prophet. Prawie wszystkie szczyty oscylacji cenowej są przeszacowane, a model MLP nie jest w stanie dokładnie przewidzieć poziomu wzrostów ani spadków cen energii elektrycznej. Wartości prognoz są znacznie bardziej rozproszone niż w przypadku modelu Prophet.

\begin{figure}[H]
    \centering
    \includegraphics[width=1.0\textwidth]{../../plots/mlp2/errors_over_time_(64, 64, 32, 16, 8)_Niestabilny_2023_skrócony.png}
    \caption{Błędy prognoz dla modelu MLP w okresie niestabilnym. Opracowanie własne.}
    \label{fig:mlp_errors_non_stable_period}
\end{figure}

Wykres błędów prognoz dla modelu MLP w okresie niestabilnym~\ref{fig:mlp_errors_non_stable_period} również potwierdza gorsze wyniki modelu MLP. Wartości absolutne błędów dochodzą do poziomu 400 PLN/MWh.

Histogram reszt~\ref{fig:mlp_residuals_non_stable_period} dla modelu MLP ma szczyt w okolicy zera, ale ma dłuższe ogony po obu stronach. Histogram wykazuje większe odchylenia od rozkładu normalnego. Widoczne są reszty o dużych wartościach, szczególnie po lewej stronie histogramu.

\begin{figure}[H]
    \centering
    \includegraphics[width=1.0\textwidth]{../../plots/mlp2/mlp_errors_histogram_short_unstable_(64, 64, 32, 16, 8).png}
    \caption{Histogram reszt dla modelu MLP w okresie niestabilnym. Opracowanie własne.}
    \label{fig:mlp_residuals_non_stable_period}
\end{figure}

    \chapter{Podsumowanie wyników i wnioski}
\label{ch:podsumowanie}



    % Bibliografia - musi być
    % Bibliography - must exist
    \bibliografia

    % Strony końcowe - można zakomentować, jeśli zbędne
    % Additional pages - comment out if not needed
    
    % Wykaz symboli i skrótów - patrz opis w tekście przykładowym
    \acronymslist

    % Spis rysunków
    \listoffigures
    % Spis tabel
    \listoftables
    % Załączniki (plik appendices.tex)
    \easyappendices
\end{document}
%%%%%%%%%%%%%%%%%%%%%%%%%%%%%%%%%%%%%%%%%%%%%%%%%%%%%%%%%%%%%%%%%%%%%%%%%%%

